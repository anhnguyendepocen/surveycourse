\documentclass[compress, 12pt]{beamer} %Makes presentation

%\documentclass[handout]{beamer} %Makes Handouts
\usetheme{Singapore} %Gray with fade at top
\useoutertheme[subsection=false]{miniframes} %Supppress subsection in header
\useinnertheme{rectangles} %Itemize/Enumerate boxes
\usecolortheme{seagull} %Color theme
\usecolortheme{rose} %Inner color theme

\definecolor{light-gray}{gray}{0.75}
\definecolor{dark-gray}{gray}{0.55}
\setbeamercolor{item}{fg=light-gray}
\setbeamercolor{enumerate item}{fg=dark-gray}

\setbeamertemplate{navigation symbols}{}
\setbeamertemplate{mini frames}[default]
\setbeamercovered{dynamics}
\setbeamerfont*{title}{size=\Large,series=\bfseries}

%\setbeameroption{notes on second screen} %Dual-Screen Notes
%\setbeameroption{show only notes} %Notes Output

\setbeamertemplate{frametitle}{\vspace{.5em}\bfseries\insertframetitle}
\newcommand{\heading}[1]{\noindent \textbf{#1}\\ \vspace{1em}}

\usepackage{bbding,color,multirow,times,ccaption,tabularx,graphicx,verbatim,booktabs,fixltx2e}
\usepackage{colortbl} %Table overlays
\usepackage[english]{babel}
\usepackage[latin1]{inputenc}
\usepackage[T1]{fontenc}

%\author[]{Thomas J. Leeper}
\institute[]{
  \inst{}%
  Department of Political Science and Government\\Aarhus University
}


\title{Sampling Techniques and Questionnaire Design}

\date[]{September 29, 2014}

\begin{document}

\frame{\titlepage}

\frame{\tableofcontents}

\section{Stratified Sampling}
\frame{\tableofcontents[currentsection]}

\frame{
	\frametitle{Review: Stratified Sampling}
	\begin{itemize}\itemsep2em
		\item What is it?
		\item Why do we do it?
		\item Most useful when subpopulations are:
		    \begin{enumerate}
		    \item identifiable in advance
		    \item differ from one another
		    \item have low within-stratum variance
		    \end{enumerate}
	\end{itemize}
}

\frame{
	\frametitle{Review: Outline of Process}
	\begin{enumerate}\itemsep1em
		\item Identify our population
		\item Construct a sampling frame
		\item Identify variables we already have that are related to our survey variables of interest
		\item Stratify or subset or sampling frame based on these characteristics
		\item Collect an SRS (of some size) within each stratum
		\item Aggregate our results
	\end{enumerate}
}

\frame{
	\frametitle{Review: Estimates from a stratified sample}
	\begin{itemize}\itemsep2em
		\item Within-strata estimates are calculated just like an SRS
		\item Within-strata variances are calculated just like an SRS
		\vspace{1em}
		\item Sample-level estimates are weighted averages of stratum-specific estimates
		\item Sample-level variances are weighted averages of strataum-specific variances
	\end{itemize}
}


\frame{
	\frametitle{Review: Design effect}
	\begin{itemize}\itemsep2em
		\item Ratio of variances in a design against a same-sized SRS
		\item $d^2 = \frac{Var_{stratified}(y)}{Var_{SRS}(y)}$
		\item Possible to convert design effect into an \textit{effective sample size}:
		\item $n_{effective} = \frac{n}{d}$
	\end{itemize}
}


\frame{
	\frametitle{Example Setup}
	\begin{itemize}\itemsep2em
		\item Interested in individual-level rate of crime victimization in Denmark
		\item We think rates differ among native-born and immigrant populations
		\item Assume immigrants make up 12\% of population
		\item Compare uncertainty from different designs ($n=1000$)
	\end{itemize}
}

\frame{
	\frametitle{SRS}
	\begin{itemize}\itemsep1em
		\item Assume equal rates across groups ($p=0.10$)
		\item Overall estimate is just $\frac{Victims}{n}$
		\item $SE(p) = \sqrt{\frac{p(1-p)}{n-1}}$
		\item $SE(p) = \sqrt{\frac{0.09}{999}} = 0.0095$
		\item<2-> SEs for subgroups (native-born and immigrants)?
		\item<3-> What happens if we don't get any immigrants in our sample?
	\end{itemize}
}


\frame{
	\frametitle{Proportionate Allocation I}
	\begin{itemize}\itemsep0.75em
		\item Assume equal rates across groups
		\item Sample 880 native-born and 120 immigrant individuals
		\item $SE(p) = \sqrt{Var(p)}$, where
		\begin{itemize}\itemsep0.75em
			\item $Var(p) = \sum_{h=1}^{H}(\frac{N_h}{N})^2 \frac{p_h(1-p_h)}{n_h - 1}$
			\item $Var(p) = (\frac{0.09}{879})(.88^2) + (\frac{0.09}{119})(.12^2)$
			\item $SE(p) = 0.0095$
		\end{itemize}
		\item Design effect: $d^2 = \frac{0.0095^2}{0.0095^2} = 1$
	\end{itemize}
}

\frame{
	\frametitle{Proportionate Allocation I}
	\begin{itemize}\itemsep0.75em
		\item Note that in this design we get different levels of uncertainty for subgroups
		\item $SE(p_{native}) = \sqrt{\frac{p(1-p)}{879}} = \sqrt{\frac{0.09}{879}} = 0.010$
		\item $SE(p_{imm}) = \sqrt{\frac{p(1-p)}{119}} = \sqrt{\frac{0.09}{119}} = 0.028$
	\end{itemize}
}



\frame{
	\frametitle{Proportionate Allocation IIa}
	\begin{itemize}\itemsep2em
		\item Assume different rates across groups (immigrants higher risk)
		\item $p_{native}=0.1$ and $p_{imm}=0.3$ (thus $p_{pop} = 0.124$)
		\item $Var(p) = \sum_{h=1}^{H}(\frac{N_h}{N})^2 \frac{p_h(1-p_h)}{n_h - 1}$
		\item $Var(p) = (\frac{0.09}{879})(.88^2) + \frac{0.21}{119})(.12^2))$
		\item $SE(p) = 0.01022$
	\end{itemize}
}

\frame{
	\frametitle{Proportionate Allocation IIa}
	\begin{itemize}\itemsep2em
		\item $SE(p) = 0.01022$
		\item Compare to SRS:
		\begin{itemize}
			\item $SE(p) = \sqrt{\frac{0.124(1-0.124)}{n-1}} = 0.0104$
		\end{itemize}
		\item Design effect: $d^2 = \frac{0.01022^2}{0.0104^2} = 0.9657$
		\item $n_{effective} = \frac{n}{sqrt(d^2)} = 1017$
	\end{itemize}
}

\frame{
	\frametitle{Proportionate Allocation IIa}
	\begin{itemize}\itemsep2em
		\item Subgroup variances are still different
		\item $SE(p_{native}) = \sqrt{\frac{p(1-p)}{879}} = \sqrt{\frac{.09}{879}} = 0.010$
		\item $SE(p_{imm}) = \sqrt{\frac{p(1-p)}{119}} = sqrt{\frac{.21}{119}} = 0.040$
	\end{itemize}
}


\frame{
	\frametitle{Proportionate Allocation IIb}
	\begin{itemize}\itemsep2em
		\item Assume different rates across groups (immigrants lower risk)
		\item $p_{native}=0.3$ and $p_{imm}=0.1$ (thus $p_{pop} = 0.276$)
		\item $Var(p) = \sum_{h=1}^{H}(\frac{N_h}{N})^2 \frac{p_h(1-p_h)}{n_h - 1}$
		\item $Var(p) = (\frac{0.21}{879})(.88^2) + \frac{0.09}{119})(.12^2))$
		\item $SE(p) = 0.014$
	\end{itemize}
}

\frame{
	\frametitle{Proportionate Allocation IIb}
	\begin{itemize}\itemsep2em
		\item $SE(p) = 0.014$
		\item Compare to SRS:
		\begin{itemize}
			\item $SE(p) = \sqrt{\frac{0.276(1-0.276)}{n-1}} = 0.0141$
		\end{itemize}
		\item Design effect: $d^2 = \frac{0.014^2}{0.0141^2} = 0.9859$
		\item $n_{effective} = \frac{n}{sqrt(d^2)} = 1007$
	\end{itemize}
}

\frame{
	\frametitle{Proportionate Allocation IIb}
	\begin{itemize}\itemsep2em
		\item Subgroup variances are still different
		\item $SE(p_{native}) = \sqrt{\frac{p(1-p)}{879}} = \sqrt{\frac{.21}{879}} = 0.0155$
		\item $SE(p_{imm}) = \sqrt{\frac{p(1-p)}{119}} = sqrt{\frac{.09}{119}} = 0.0275$
	\end{itemize}
}



\frame{
	\frametitle{Proportionate Allocation IIc}
	\begin{itemize}\itemsep2em
		\item Look at same design, but a different survey variable (household size)
		\item Assume: $\bar{y}_{native}=4$ and $\bar{Y}_{imm}=6$ (thus $\bar{Y}_{pop} = 4.24$)
		\item Assume: $Var(Y_{native}) = 1$ and $Var(Y_{imm}) = 3$ and $Var(Y_{pop}) = 4$
		\item $Var(\bar{y}) = \sum_{h=1}^{H}(\frac{N_h}{N})^2 \frac{s_h^2}{n_h}$
		\item $SE(\bar{y}) = \sqrt{\frac{1^2}{880}(.88^2) + \frac{3^2}{120}(.12^2)} = 0.0443$
	\end{itemize}
}

\frame{
	\frametitle{Proportionate Allocation IIc}
	\begin{itemize}\itemsep2em
		\item $SE(\bar{y}) = 0.0443$
		\item Compare to SRS:
		\begin{itemize}
			\item $SE(\bar{y}) = \sqrt{\frac{s^2}{n}} = \sqrt{4/1000} = 0.0632$
		\end{itemize}
		\item Design effect: $d^2 = \frac{0.0443^2}{0.0632^2} = 0.491$
		\item $n_{effective} = \frac{n}{sqrt(d^2)} = 1427$
		\item<2-> Why is $d^2$ so much larger here?
	\end{itemize}
}



\frame{
	\frametitle{Disproportionate Allocation I}
	\begin{itemize}\itemsep1em
		\item Previous designs obtained different precision for subgroups
		\item Design to obtain stratum-specific precision (e.g., $SE(p_h) = 0.02$)
		\item $n_h = \frac{p(1-p)}{v(p)} = \frac{p(1-p)}{SE^2}$
		\item $n_{native} = \frac{0.09}{0.02^2} = 225$
		\item $n_{imm} = \frac{0.21}{0.02^2} = 525$
		\item $n_{total} = 225 + 525 = 750$
	\end{itemize}
}


\frame{
	\frametitle{Disproportionate Allocation II}
	\begin{itemize}\itemsep2em
		\item Neyman optimal allocation
		\item How does this work?
		\begin{itemize}
			\item Allocate cases to strata based on within-strata variance
			\item Only works for one variable at a time
			\item Need to know within-strata variance
		\end{itemize}
	\end{itemize}
}

\frame{
	\frametitle{Disproportionate Allocation II}
	\begin{itemize}\itemsep1em
		\item Assume big difference in victimization
		\item $p_{native}=0.01$ and $p_{imm}=0.50$  (thus $p_{pop} = 0.0688$)
		\item Allocate according to: $n_h = n \frac{W_h S_h}{\sum_{h=1}^{H} W_h S_h}$
		\item $\sum_{h=1}^{H} W_h S_h = (0.88 * 0.0099) + (0.12 * 0.25) = 0.0387$
		\item $n_{native} = 1000 \frac{0.0087}{0.0387} = 225$
		\item $n_{imm} = 1000 \frac{0.03}{0.0387} = 775$
	\end{itemize}
}

\frame{
	\frametitle{Disproportionate Allocation II}
	\begin{itemize}\itemsep2em
		\item $SE(p_{native}) = \sqrt{\frac{p(1-p)}{225}} = \sqrt{\frac{0.0099}{225}} = 0.00663$
		\item $SE(p_{imm}) = \sqrt{\frac{p(1-p)}{775}} = \sqrt{\frac{.25}{775}} = 0.01796$
		\item $Var(p) = \sum_{h=1}^{H}(\frac{N_h}{N})^2 \frac{p_h(1-p_h)}{n_h - 1}$
		\item $Var(p) = (\frac{0.0099}{225})(.88^2) + (\frac{0.25}{775})(.12^2)$
		\item $SE(p) = 0.00622$
	\end{itemize}
}

\frame{
	\frametitle{Disproportionate Allocation II}
	\begin{itemize}\itemsep2em
		\item $SE(p) = 0.00622$
		\item Compare to SRS:
		\begin{itemize}
			\item $SE(p) = \sqrt{\frac{0.0688(1-0.0688)}{n-1}} = 0.008$
		\end{itemize}
		\item Design effect: $d^2 = \frac{0.00622^2}{0.008^2} = 0.6045$
		\item $n_{effective} = \frac{n}{sqrt(d^2)} = 1286$
	\end{itemize}
}

\frame{
	\frametitle{Final Considerations}
	\begin{itemize}\itemsep2em
		\item Reductions in uncertainty come from creating homogeneous groups
		\item Estimates of design effects are variable-specific
		\item Sampling variance calculations do not factor in time, costs, or feasibility
	\end{itemize}
}


\frame{\frametitle{Questions about stratified sampling?}}

\section{Cluster Sampling}
\frame{\tableofcontents[currentsection]}

\frame{
	\frametitle{Cluster Sampling}
	\begin{itemize}\itemsep2em
		\item What is it?
		\item Why do we do?
		\item<2-> Most useful when:
		\begin{enumerate}
			\item Population has a clustered structure
			\item Unit-level sampling is expensive or not feasible
			\item Clusters are similar
		\end{enumerate}
	\end{itemize}
}

\frame{
	\frametitle{Cluster Sampling}
	\begin{itemize}\itemsep2em
		\item Advantages
		\begin{itemize}
			\item<2-> Cost savings!
			\item<2-> Capitalize on clustered structure
		\end{itemize}
		\item<3-> Disadvantages
		\begin{itemize}
			\item<4-> Units tend to cluster for complex reasons (self-selection)
			\item<4-> Major increase in uncertainty if clusters differ from each other
			\item<4-> Complex to design (and possibly to administer)
			\item<4-> Analysis is much more complex than SRS or stratified sample
		\end{itemize}
	\end{itemize}
}


\frame{
	\frametitle{Cluster Sampling}
	\begin{itemize}\itemsep2em
		\item Number of stages
			\begin{itemize}
				\item One-stage sampling
				\item Two- or more-stage sampling
			\end{itemize}
		\item Number of clusters
		\item Sample size w/in clusters
		\item Everything depends on variability of clusters
	\end{itemize}
}


\frame{
	\frametitle{Sampling Variance for Cluster Sampling}
	\begin{itemize}\itemsep2em
		\item Sampling variance depends on \textit{between}-cluster variation:\\
		$Var(\bar{y}) = (\frac{1-f}{a})(\frac{1}{a-1})(\sum_{\alpha=1}^{a}(\bar{y}_{\alpha} - \bar{y})^2)$
		\item When \textit{between}-cluster variance is high, \textit{within}-cluster variance is likely to be low
			\begin{itemize}
				\item ``Cluster homogeneity''
			\end{itemize}
	\end{itemize}
}

\frame{
	\frametitle{Design Effect for Cluster Sampling}
	\begin{itemize}\itemsep2em
		\item Cluster samples almost always less \textit{statistically} efficient than SRS
		\item Design Effect depends on cluster homogeneity:
		\begin{itemize}\itemsep1em
			\item $d^2 = \frac{Var_{clustered}(y)}{Var_{SRS}(y)}$
			\item $d^2 = 1 + (n_{cluster}-1)roh$
		\end{itemize}
		\item \textit{roh} (\textit{intraclass correlation coefficient}):
		\begin{itemize}
			\item Proportion of unit-level variance that is between-clusters
			\item Generally positive and small (about 0.00 to 0.10)
		\end{itemize}
	\end{itemize}
}

\frame{\frametitle{Questions about cluster sampling?}}

\frame{
	\frametitle{Example: Burnham et al.}
	\begin{itemize}\itemsep2em
		\item What is the research question?
		\item<2-> What are the population and unit of analysis?
		\item<3-> What is the sampling strategy? Why?
		\item<4-> What do they find?
	\end{itemize}
}

\frame{
	\frametitle{Complex Survey Designs}
	\begin{itemize}\itemsep2em
		\item Often stratification and clustering are used together
		\item The choice of design must do at least one of:\\
		\begin{itemize}
			\item Improve statistical efficiency
			\item Improve ease/cost of implementation
		\end{itemize}
		\item Design effects
		\item Weights
	\end{itemize}
}



\section{Questionnaire Design}
\frame{\tableofcontents[currentsection]}

\frame{
	\frametitle{Concept definition and Operationalization}
	\begin{itemize}\itemsep2em
		\item Questionnaires start with concept definition
		\item Multiple ways to operationalize any concept
		\item Important concepts may require multiple measures
	\end{itemize}
}

\frame{
	\frametitle{Topics of questions}
	\begin{itemize}\itemsep2em
		\item Evaluations (opinions, attitudes, etc.)
		\item Recall (behavior, events, knowledge, etc.)
		\begin{itemize}
		    \item Demographics (age, sex, ethnicity, etc.)
		\end{itemize}
	\end{itemize}
}

\frame{
	\frametitle{Structure of a question}
	\begin{itemize}\itemsep1em
		\item Survey mode
		\item Survey context
		\item Vignette or introductory text
		\item Question itself
		\item Response format and options
		\item Follow-ups, branches, checks, validation, clarification
	\end{itemize}
}


\frame{
	\frametitle{Evaluative questions}
	\begin{itemize}\itemsep2em
		\item Name an object of evaluation
		\item Possibly describe that object
		\item Ask for a transformation of the evaluation onto a set of responses
		\item Individuals differ in how they form opinions
		    \begin{itemize}
		        \item Memory-based processing
		        \item Online processing
		    \end{itemize}
	\end{itemize}
}

\frame{
	\frametitle{Response options for evaluative questions}
	\begin{itemize}\itemsep1em
		\item Ratings
			\begin{itemize}
				\item Bipolar
				\item Branching
				\item Unipolar
			\end{itemize}
		\item Scales/Thermometers
		\item Agree-disagree
		\item Forced choices
		\item Open-ended
		\item Rankings (note: need alternatives to rank against)
	\end{itemize}
}

\frame{
	\frametitle{Extended Example}
	\begin{itemize}\itemsep2em
		\item Public opinion survey in Denmark
		\item Construct: Opinion toward Danish involvement in air strikes on Islamic State militants in Iraq and Syria
	\end{itemize}
}

\frame{
	\frametitle{Example: Rating (bipolar)}
	Do you support or oppose Denmark's participation in U.S.-led air strikes on Islamic State (IS) in Iraq and Syria?
	\begin{itemize}
		\item Strongly support
		\item Somewhat support
		\item Neither support nor oppose
		\item Somewhat oppose
		\item Strongly oppose
	\end{itemize}
}

\frame{
	\frametitle{Example: Rating (branching)}
	Do you support or oppose Denmark's participation in U.S.-led air strikes on Islamic State (IS) in Iraq and Syria?
	\begin{itemize}
		\item Support
		\item Neither support nor oppose
		\item Oppose
	\end{itemize}
	\vspace{1em}
	Would you say that you strongly [support|oppose] or somewhat [support|oppose] Denmark's participation?
	\begin{itemize}
		\item Strongly
		\item Somewhat
	\end{itemize}
}

\frame{
	\frametitle{Example: Rating (bipolar)}
	Are you favorable or unfavorable toward Denmark's participation in U.S.-led air strikes on Islamic State (IS) in Iraq and Syria?
	\begin{itemize}
		\item Very favorable
		\item Somewhat favorable
		\item Neither favorable nor unfavorable
		\item Somewhat unfavorable
		\item Strongly unfavorable
	\end{itemize}
}

\frame{
	\frametitle{Example: Rating (unipolar)}
	To what extent do you support Denmark's participation in U.S.-led air strikes on Islamic State (IS) in Iraq and Syria?
	\begin{itemize}
		\item Strongly
		\item Moderately
		\item Somewhat
		\item Not at all
	\end{itemize}
}

\frame{
	\frametitle{Example: Rating (unipolar)}
	How favorable are you toward Denmark's participation in U.S.-led air strikes on Islamic State (IS) in Iraq and Syria?
	\begin{itemize}
		\item Extremely favorable
		\item Very favorable
		\item Moderately favorable
		\item Somewhat favorable
		\item Not at all favorable
	\end{itemize}
}

\frame{
	\frametitle{Example: Numbered Scale}
	On a scale from 1 to 5, with 1 being ``strongly oppose'' and 5 being ``strongly support,'' to what extent do you support Denmark's participation in U.S.-led air strikes on Islamic State (IS) in Iraq and Syria?
	\begin{enumerate}
		\item Strongly oppose
		\item 
		\item 
		\item 
		\item Strongly support
	\end{enumerate}
}

\frame{
	\frametitle{Example: Thermometer}
	We would like to get your feelings toward some of political policies. Please rate your support for the policy using something we call the feeling thermometer. Ratings between 50 degrees and 100 degrees mean that you feel favorable and warm toward the policy. Ratings between 0 degrees and 50 degrees mean that you don't feel favorable toward the policy. You would rate the policy at the 50 degree mark if you don't feel particularly favorable or unfavorable toward.
	
	\vspace{1em}
	Denmark's participation in U.S.-led air strikes on Islamic State (IS) in Iraq and Syria.
	\begin{itemize}
		\item 0--100 slider
	\end{itemize}
}


\frame{
	\frametitle{Example: Agree/Disagree (bipolar)}
	To what extent do you agree with the following statement: I support Denmark's participation in U.S.-led air strikes on Islamic State (IS) in Iraq and Syria.
	\begin{itemize}
		\item Strongly agree
		\item Somewhat agree
		\item Neither agree nor disagree
		\item Somewhat disagree
		\item Strongly disagree
	\end{itemize}
}

\frame{
	\frametitle{Example: Agree/Disagree (unipolar)}
	To what extent do you agree with the following statement: I support Denmark's participation in U.S.-led air strikes on Islamic State (IS) in Iraq and Syria.
	\begin{itemize}
		\item Agree completely
		\item Agree to a large extent
		\item Agree to a moderate extent
		\item Agree a little bit
		\item Agree not at all
	\end{itemize}
}

\frame{
	\frametitle{Example: Forced choice}
	When thinking about Denmark's participation in U.S.-led air strikes on Islamic State (IS) in Iraq and Syria, which of the following comes closer to your opinion:
	\begin{itemize}
		\item Denmark should participate in air strikes
		\item Denmark should not participate in air strikes
	\end{itemize}
}

\frame{
	\frametitle{Example: Open-ended}
	In your own words, how would you describe your opinion on Denmark's participation in U.S.-led air strikes on Islamic State (IS) in Iraq and Syria?
}


\frame{
	\frametitle{Additional Considerations}
	\begin{itemize}\itemsep1em
		\item How many response categories?
		\item Middle category (presence and label)
		\item ``no opinion'' and/or ``don't know'' options
		\item Probe if ``no opinion'' or ``don't know''?
		    \begin{itemize}
                \item Encourage guessing?
    		    \item Clarify/describe object of evaluation?
		    \end{itemize}
		\item Branching format?
		\item Order of response categories
		\item Changes based on survey mode
	\end{itemize}
}


\frame{\frametitle{Questions about writing evaluative questions?}}


\frame{
	\frametitle{Activity!}
	\begin{itemize}\itemsep2em
		\item Generate questions in pairs
		\item Discuss with the class
	\end{itemize}
}



\section{Preview of Next Week}
\frame{\tableofcontents[currentsection]}

\frame{
	\frametitle{Assignment for next week}
	\begin{itemize}\itemsep1em
		\item What constructs/concepts do you intend to measure in your survey?
		\item How do you plan to measure these?
		\item How have these constructs been operationalized in other research?
	\end{itemize}
}

\frame{
	\frametitle{Next week's agenda}
	\begin{itemize}\itemsep1em
		\item Continue questionnaire design
		\item Measuring sensitive information
		\item Measuring knowledge
		\item Reference periods
	\end{itemize}
}


\appendix
\frame{}

\end{document}
