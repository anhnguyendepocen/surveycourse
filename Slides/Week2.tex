\documentclass[compress, 12pt]{beamer} %Makes presentation

%\documentclass[handout]{beamer} %Makes Handouts
\usetheme{Singapore} %Gray with fade at top
\useoutertheme[subsection=false]{miniframes} %Supppress subsection in header
\useinnertheme{rectangles} %Itemize/Enumerate boxes
\usecolortheme{seagull} %Color theme
\usecolortheme{rose} %Inner color theme

\definecolor{light-gray}{gray}{0.75}
\definecolor{dark-gray}{gray}{0.55}
\setbeamercolor{item}{fg=light-gray}
\setbeamercolor{enumerate item}{fg=dark-gray}

\setbeamertemplate{navigation symbols}{}
\setbeamertemplate{mini frames}[default]
\setbeamercovered{dynamics}
\setbeamerfont*{title}{size=\Large,series=\bfseries}

%\setbeameroption{notes on second screen} %Dual-Screen Notes
%\setbeameroption{show only notes} %Notes Output

\setbeamertemplate{frametitle}{\vspace{.5em}\bfseries\insertframetitle}
\newcommand{\heading}[1]{\noindent \textbf{#1}\\ \vspace{1em}}

\usepackage{bbding,color,multirow,times,ccaption,tabularx,graphicx,verbatim,booktabs,fixltx2e}
\usepackage{colortbl} %Table overlays
\usepackage[english]{babel}
\usepackage[latin1]{inputenc}
\usepackage[T1]{fontenc}

%\author[]{Thomas J. Leeper}
\institute[]{
  \inst{}%
  Department of Political Science and Government\\Aarhus University
}


\title{Who is being surveyed?}

\date[]{September 15, 2014}

\begin{document}

\frame{\titlepage}

\frame{\tableofcontents}

\section{Review of Last Week}
\frame{\tableofcontents[currentsection, hideothersubsections,]}

\frame{
	\frametitle{Paul Lazarsfeld (1901--1976)}
	\begin{itemize}\itemsep1em
		\item Originally from Austria; spent career at Columbia University
		\item Pioneered the study of mass media (Princeton Radio Project)
		\begin{itemize}
			\item \textit{The War of the Worlds} (1938)
		\end{itemize}
		\item Created the survey panel to study radio impact
		\item First ever election surveys: \textit{The People's Choice} (1940) and \textit{Voting} (1948)
		\item Two-step flow of influence: \textit{Personal Influence}
	\end{itemize}
}


\frame{
	\frametitle{Criteria for Causal Inference}
	\begin{enumerate}\itemsep1em
		\item Relationship
		\item Temporal precedence
		\item Nonconfounding
		\item Mechanism
		\item Level of analysis
	\end{enumerate}
}


\frame{
	\frametitle{Assignment for this week}
	\begin{enumerate}\itemsep1em
		\item Form groups of 3 (or so)
		\item Present your research question idea(s)
		\item Give feedback to your peers on the idea
		\item Share some with the whole class
	\end{enumerate}
}


\section{New Material to Cover}
\frame{\tableofcontents[currentsection]}


\subsection{Total Survey Error}
\frame{\tableofcontents[currentsubsection, subsectionstyle=show/shaded, sectionstyle=show/hide]}
 
\frame{
	\frametitle{Total Survey Error}
	\begin{itemize}\itemsep2em
		\item<1-> Envision the perfect survey!
		\item Errors introduced in design, implementation, and analysis
		\item Late 20th-century survey research focused on minimizing particular sources of error
		\item ``Total Survey Error'' approach is about trade-offs between all sources of error, costs, and time
	\end{itemize}
}

 
\subsection{Populations}
\frame{\tableofcontents[currentsubsection, subsectionstyle=show/shaded, sectionstyle=show/hide]}

\frame{
	\frametitle{Inference Population}
	\begin{itemize}\itemsep2em
		\item We want to speak to a population
		\item But what population is it?
		\item<2-> Example: ``The Danish population''
	\end{itemize}
}

\frame{
	\frametitle{Population Census}
	\begin{itemize}\itemsep2em
		\item All population units are in study
		\item<2-> History of national censuses
			\begin{itemize}
				\item<2-> Denmark 1769--1970 (sporadic)
				\item<2-> U.S. 1790 (decennial)
				\item<2-> India 1871 (decennial)
			\end{itemize}
		\item<3-> Other kinds of census
			\begin{itemize}
				\item<3-> Citizen registry
				\item<3-> Commercial, medical, government records
				\item<3-> ``Big data''
			\end{itemize}
	\end{itemize}
}

\frame{
	\frametitle{Advantages and Disadvantages}
	\begin{itemize}\itemsep2em
		\item Advantages
			\begin{itemize}
				\item<2-> Perfectly representative
				\item<2-> Sample statistics are population parameters
			\end{itemize}
		\item Disadvantages
			\begin{itemize}
				\item<3-> Costs
				\item<3-> Feasibility
				\item<3-> Need
			\end{itemize}		
	\end{itemize}
}


\subsection{Representativeness}
\frame{\tableofcontents[currentsubsection, subsectionstyle=show/shaded, sectionstyle=show/hide]}

\frame{
	\frametitle{Representativeness}
	\begin{itemize}\itemsep2em
		\item What does it mean for a sample to be representative?
	\end{itemize}
}


\frame{
	\frametitle{Obtaining Representativeness}
	\begin{itemize}\itemsep2em
		\item<1-> Quota sampling (common prior to the 1940s)
		\item<2-> Simple random sampling
		\item<3-> Advanced survey designs (discuss next week)
	\end{itemize}
}

\frame{
	\frametitle{Convenience Samples}
	\begin{itemize}\itemsep2em
		\item<1-> What is a convenience sample?
		\item<2-> Different types:
			\begin{itemize}
				\item<2-> Passive/opt-in
				\item<2-> Sample of convenience (not a sample per se)
				\item<2-> Sample matching
				\item<2-> Online panels
			\end{itemize}
		\item<3-> ``Purposive'' samples (common in qualitative studies)
	\end{itemize}
}

\frame{\frametitle{Questions about convenience samples?}}


\subsection{Sampling Frames}
\frame{\tableofcontents[currentsubsection, subsectionstyle=show/shaded, sectionstyle=show/hide]}

\frame{
	\frametitle{Sampling Frames}
	\begin{itemize}\itemsep2em
		\item Enumeration (listing) of all units eligible for sample selection
		\item Two flavors:
			\begin{itemize}
				\item Random sample from an ordered list
				\item Systematic sampling from a randomized list
			\end{itemize}
		\item Building a sampling frame
			\begin{itemize}
				\item Combine existing lists
				\item Canvass/enumerate from scratch
			\end{itemize}
	\end{itemize}
}

\frame{
	\frametitle{Big considerations}
	\begin{itemize}\itemsep2em
		\item Coverage!
			\begin{itemize}
				\item Undercoverage
				\item Overcoverage
			\end{itemize}
		\item<2-> What is a unit?
		\item<3-> Clustering
		\item<4-> Overlap between units
		\item<5-> List maintenance
	\end{itemize}
}

\frame{
	\frametitle{Multi-frame Designs}
	\begin{itemize}\itemsep2em
		\item Construct one sample from multiple sampling frames
		\item E.g., ``Dual-frame'' (landline and mobile)
		\item Analytically complicated
			\begin{itemize}
				\item Overlap of frames
				\item Sample probabilities in each frame
			\end{itemize} 
	\end{itemize}
}


\subsection{Sampling without a Frame}
\frame{\tableofcontents[currentsubsection, subsectionstyle=show/shaded, sectionstyle=show/hide]}

\frame{
	\frametitle{Sampling without a Sampling Frame}
	\begin{itemize}\itemsep2em
		\item Sometimes we have a population that can be sampled but not (easily) enumerated in full
		\item<2-> Examples
			\begin{itemize}
				\item<2-> Protest attendees
				\item<3-> Streams (e.g., people buying groceries)
				\item<4-> Points in time
			\end{itemize}
		\item<5-> Population is the sampling frame
	\end{itemize}
}

\frame{
	\frametitle{Rare or ``hidden'' populations}
	\begin{itemize}\itemsep2em
		\item Big concern: coverage!
		\item<2-> Solutions?
			\begin{itemize}
				\item<3-> Snowball sampling
				\item<3-> Informant sampling
				\item<3-> Targeted sampling
				\item<3-> Respondent-driven sampling
			\end{itemize}
		\item<3-> How does RDS work?
	\end{itemize}
}


\frame{\frametitle{Questions?}}


\frame{
	\frametitle{Activity!}
	\begin{itemize}\itemsep2em
		\item Work in pairs
		\item Develop two sampling frames/sampling strategies for a population
		\item Share with class and discuss
	\end{itemize}
}

\subsection{Simple Random Sampling}
\frame{\tableofcontents[currentsubsection, subsectionstyle=show/shaded, sectionstyle=show/hide]}

\frame{
	\frametitle{Simple Random Sampling (SRS)}
	\begin{itemize}\itemsep2em
		\item Advantages
			\begin{itemize}
				\item Simplicity of sampling
				\item Simplicity of analysis
			\end{itemize}
		\item Disadvantages
			\begin{itemize}
				\item Need sampling frame and units without any structure
				\item Possibly expensive
			\end{itemize}
	\end{itemize}
}

\frame{
	\frametitle{Sample Estimates from an SRS}
	\begin{itemize}\itemsep2em
		\item Each unit in frame has equal probability of selection
		\item Sample statistics are unweighted
		\item Variances are easy to calculate
		\item Easy to calculate sample size need for a particular variance
	\end{itemize}
}

\frame{
	\frametitle{Sample mean}
	\begin{equation}
	\bar{y} = \frac{1}{n}\sum_{i=1}^{n}y_i
	\end{equation}
	where $y_i = $ value for a unit, and\\
	$n = $ sample size
	
	\begin{equation}
	SE_{\bar{y}} = \sqrt{(1-f)\frac{s^2}{n}}
	\end{equation}
	where $f = $ proportion of population sampled,\\
	$s^2 = $ sample variance, and\\
	$n = $ sample size
}

\frame{
	\frametitle{Sample proportion}
	\begin{equation}
	\bar{y} = \frac{1}{n}\sum_{i=1}^{n}y_i
	\end{equation}
	where $y_i = $ value for a unit, and\\
	$n = $ sample size
	
	\begin{equation}
	SE_{\bar{y}} = \sqrt{\frac{(1-f)}{(n-1)}p(1-p)}
	\end{equation}
	where $f = $ proportion of population sampled,\\
	$p = $ sample proportion, and\\
	$n = $ sample size
}

\frame{
	\frametitle{Estimating sample size}
	\begin{itemize}
		\item Imagine we want to conduct a political poll
		\item We want to know what percentage of the public will vote for which coalition/party
		\item How big of a sample do we need to make a relatively precise estimate of voter support?
	\end{itemize}
}

\frame{
	\frametitle{Estimating sample size}
	\begin{equation}
		Var(p) = (1-f)\frac{p(1-p)}{n-1}
	\end{equation}
	
	Given the large population:	
	\begin{equation}
		Var(p) = \frac{p(1-p)}{n-1}
	\end{equation}
	
	\vspace{1em}
	Need to solve the above for $n$.
	\begin{equation}
	\only<2->{n = \frac{p(1-p)}{v(p)} = \frac{p(1-p)}{SE^2}}
	\end{equation}
}

\frame{
	\frametitle{Estimating sample size}
    Determining sample size requires:
    	\begin{itemize}
    		\item A possible value of $p$
    		\item A desired precision (SE)
    	\end{itemize}
	\vspace{1em}
	If support for each coalition is evenly matched ($p = 0.5$):
	\begin{equation}
	n = \frac{0.5(1-0.5)}{SE^2} = \frac{0.25}{SE^2}
	\end{equation}
}

\frame{
	\frametitle{Estimating sample size}
	What precision (margin of error) do we want?
	\begin{itemize}
		\item +/- 2 percentage points: $SE = 0.01$
			\begin{equation}
			n = \frac{0.25}{0.01^2} = \frac{0.25}{0.0001} = 2500
			\end{equation}
		\item<2-> +/- 5 percentage points: $SE = 0.025$
			\begin{equation}
			n = \frac{0.25}{0.000625} = 400
			\end{equation}
		\item<3-> +/- 0.5 percentage points: $SE = 0.0025$
			\begin{equation}
			n = \frac{0.25}{0.00000625} = 40,000
			\end{equation}
	\end{itemize}
}

\frame{
	\frametitle{Important considerations}
	\begin{itemize}\itemsep1em
		\item Required sample size depends on $p$ and $SE$
		\item<2-> In large populations, population size is irrelevant
		\item<3-> In small populations, precision is influenced by the proportion of population sampled
		\item<4-> In anything other than an SRS, sample size calculation is more difficult
		\item<5-> Much political science research assumes SRS even though a more complex design is actually used
	\end{itemize}
}

\frame{\frametitle{Questions about SRS?}}

\section{Preview of Next Week}
\frame{\tableofcontents[currentsection, hideothersubsections]}

\frame{
	\frametitle{Next week's agenda}
	\begin{itemize}\itemsep1em
		\item Stratified sampling
		\item Cluster sampling
		\item Estimates, variances, and design effects
		\item Discuss sampling schemes in published research
	\end{itemize}
}

\frame{
	\frametitle{Presentations?}
	\begin{itemize}\itemsep1em
		\item Burnham et al.: Mortality in Iraq
		\item Reinisch et al.: Registry data study
		\item Walker and Enticott: Surveying public managers
	\end{itemize}
}

\frame{
	\frametitle{Assignment for next week}
	\begin{itemize}\itemsep1em
		\item Find a real survey or published study based on a survey
		\item Figure out its population, sampling frame, and sample
		\item Write up 0.5-1.0 pages discussing and evaluating its sampling approach
		\item We will discuss these in class next week
	\end{itemize}
}


\appendix
\frame{}

\end{document}
