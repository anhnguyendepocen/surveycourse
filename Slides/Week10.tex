\documentclass[compress, 12pt]{beamer} %Makes presentation

%\documentclass[handout]{beamer} %Makes Handouts
\usetheme{Singapore} %Gray with fade at top
\useoutertheme[subsection=false]{miniframes} %Supppress subsection in header
\useinnertheme{rectangles} %Itemize/Enumerate boxes
\usecolortheme{seagull} %Color theme
\usecolortheme{rose} %Inner color theme

\definecolor{light-gray}{gray}{0.75}
\definecolor{dark-gray}{gray}{0.55}
\setbeamercolor{item}{fg=light-gray}
\setbeamercolor{enumerate item}{fg=dark-gray}

\setbeamertemplate{navigation symbols}{}
\setbeamertemplate{mini frames}[default]
\setbeamercovered{dynamics}
\setbeamerfont*{title}{size=\Large,series=\bfseries}

%\setbeameroption{notes on second screen} %Dual-Screen Notes
%\setbeameroption{show only notes} %Notes Output

\setbeamertemplate{frametitle}{\vspace{.5em}\bfseries\insertframetitle}
\newcommand{\heading}[1]{\noindent \textbf{#1}\\ \vspace{1em}}

\usepackage{bbding,color,multirow,times,ccaption,tabularx,graphicx,verbatim,booktabs,fixltx2e}
\usepackage{colortbl} %Table overlays
\usepackage[english]{babel}
\usepackage[latin1]{inputenc}
\usepackage[T1]{fontenc}

%\author[]{Thomas J. Leeper}
\institute[]{
  \inst{}%
  Department of Political Science and Government\\Aarhus University
}


\title{Fielding and Recruitment}

\date[]{November 17, 2014}

\begin{document}

\frame{\titlepage}

\frame{\tableofcontents}


\section{Pilot Testing}
\frame{\tableofcontents[currentsection]}

\frame{
	\frametitle{Pilot Testing}
	\begin{itemize}\itemsep2em
		\item Has anyone tried pilot testing yet?
		\item What have been your experiences so far?
	\end{itemize}
}

\section{Recruitment}
\frame{\tableofcontents[currentsection]}

\frame{
	\frametitle{Recruitment}
	\begin{itemize}\itemsep1em
		\item Often tied to mode
		\item Not always the same as mode
	\end{itemize}
}

\frame{
	\frametitle{Methods of Recruitment}
	\begin{itemize}\itemsep2em
		\item Postal mail: Letter, postcard, etc.
		\item Telephone call
		\item Email
		\item Opt-in recruitment (ads, posters, etc.)
	\end{itemize}
}

\frame{
	\frametitle{Think--Pair--Share: Recruitment}
	\begin{itemize}\itemsep2em
		\item How can we encourage participation?
		\item What are the advantages and disadvantages of different recruitment techniques and methods?
	\end{itemize}
}



\section{Response Rates}
\frame{\tableofcontents[currentsection]}

\frame{
    \frametitle{Response Rates}
    \begin{itemize}
        \item Why do we care?
        \item<2-> Survey Error
            \begin{itemize}
                \item<2-> Variance
                \item<2-> Bias
            \end{itemize}
        \item<3-> Sample size calculations (and design effects) are based on completed interviews
        \item<4-> Cost, time, and effort
    \end{itemize}
}

\frame{
    \frametitle{Response Rates}
    \begin{itemize}
        \item Imagine we need $n=1000$
        \item How many attempts to obtain that sample:\\
        \vspace{1em}
        \begin{tabular}{l r}\toprule
        Response Rate & Needed Attempts\\ \midrule
        1.00 & 1000 \\
        0.75 & 1333 \\
        0.50 & 2000 \\
        0.25 & 4000 \\
        0.10 & 10,000 \\ \bottomrule
        \end{tabular}
    \end{itemize}
}


\frame{
	\frametitle{Response Rate}
	\begin{itemize}\itemsep1em
		\item Interviews divided by eligibles
		\item $RR = \frac{I}{E}$
		\item Challenges
    		\begin{itemize}
        		\item Unknown eligibility
        		\item Partial interviews
        		\item Non-probability samples
        		\item Complex survey designs
    		\end{itemize}
        \item Cooperation Rate (I's divided by contacts)
	\end{itemize}
}

\frame{
	\frametitle{Disposition Codes}
	\begin{itemize}\itemsep2em
		\item Interviews
		\item Refusals
		\item Unknowns
		\item Ineligibles
	\end{itemize}
}

\frame{
	\frametitle{Disposition Codes}
	\begin{itemize}\itemsep1em
		\item Complete Interview (I)
		\item Partial Interview (P)
		\item Non-interviews
    		\begin{itemize}
        		\item Refusal (R)
        		\item Non-contact (NC)
        		\item Other (O)
    		\end{itemize}
		\item Unknowns (U)
		\item Ineligibles
	\end{itemize}
}

\frame{
	\frametitle{Response Rate 1\footnote{Note: Simplified slightly}}
	\Large
	\begin{itemize}\itemsep2em
		\item $RR1 = \frac{I}{(I + P) + (R + NC) + U}$
	\end{itemize}
}

\frame{
	\frametitle{Response Rate 2\footnote{Note: Simplified slightly}}
    \Large
	\begin{itemize}\itemsep2em
		\item $RR2 = \frac{I + P}{(I + P) + (R + NC) + U}$
	\end{itemize}
}

\frame{
	\frametitle{Response Rates 3 and 4\footnote{Note: Simplified slightly}}
	\Large
    \begin{itemize}\itemsep2em
		\item $RR3 = \frac{I}{(I + P) + (R + NC) + (e*U)}$
		\item $RR4 = \frac{I + P}{(I + P) + (R + NC) + (e*U)}$
		\vspace{1em}
		\item $e$ is estimated proportion eligible among unknowns
	\end{itemize}
}

\frame{
	\frametitle{Cooperation Rates}
	\Large
    \begin{itemize}\itemsep2em
		\item $COOP1 = \frac{I}{(I + P) + R}$
		\item $COOP2 = \frac{I + P}{(I + P) + R}$
	\end{itemize}
}

\frame{
	\frametitle{Refusal Rates}
	\begin{itemize}\itemsep2em
		\item Related to response rate
		\item Numerator is refusals
		\item E.g., {\Large $REF1 = \frac{I}{(I + P) + (R + NC) + U}$}
	\end{itemize}
}


\frame{
	\frametitle{Complex Survey Designs}
	\begin{itemize}\itemsep2em
		\item Stratified Sampling
    		\begin{itemize}
        		\item Sums of codes weighted by $\frac{1}{p}$
        		\item $p$ is probability of selection
    		\end{itemize}
    	\item Multi-stage sampling
        	\begin{itemize}
            	\item E.g., cluster sampling
            	\item RR is product of cluster cooperation and within-cluster response rate
        	\end{itemize}
	\end{itemize}
}


\frame{
	\frametitle{Internet Surveys}
	\begin{itemize}\itemsep2em
		\item For \textit{probability-based samples}, RR is a product of:
    		\begin{itemize}
        		\item Recruitment Rate (RR for panel enrollment)
        		\item Completion Rate (RR for specific survey)
        		\item Profile Rate (in some cases)
        		\item E.g., if Recruitment Rate is 30\% and Completion Rate is 80\%, $RR = 0.3 * 0.8 =$ 24\%
    		\end{itemize}
    	\item For \textit{non-probability samples}, RR is undefined
        	\begin{itemize}
            	\item No sampling involved (so no denominator)
            	\item If from panel, report Completion Rate
            	\item If fully opt-in, there's nothing you can do
        	\end{itemize}
	\end{itemize}
}


\section{Fielding}
\frame{\tableofcontents[currentsection]}




\section{Preview of Next Time}
\frame{\tableofcontents[currentsection]}

\frame{
	\frametitle{Agenda for next class}
	\begin{itemize}\itemsep2em
		\item Data management and codebooks
		\item Missing data and imputation
		\item Weighting
	\end{itemize}
}

\appendix
\frame{}

\end{document}
