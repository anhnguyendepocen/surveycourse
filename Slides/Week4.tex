\documentclass[compress, 12pt]{beamer} %Makes presentation

%\documentclass[handout]{beamer} %Makes Handouts
\usetheme{Singapore} %Gray with fade at top
\useoutertheme[subsection=false]{miniframes} %Supppress subsection in header
\useinnertheme{rectangles} %Itemize/Enumerate boxes
\usecolortheme{seagull} %Color theme
\usecolortheme{rose} %Inner color theme

\definecolor{light-gray}{gray}{0.75}
\definecolor{dark-gray}{gray}{0.55}
\setbeamercolor{item}{fg=light-gray}
\setbeamercolor{enumerate item}{fg=dark-gray}

\setbeamertemplate{navigation symbols}{}
\setbeamertemplate{mini frames}[default]
\setbeamercovered{dynamics}
\setbeamerfont*{title}{size=\Large,series=\bfseries}

%\setbeameroption{notes on second screen} %Dual-Screen Notes
%\setbeameroption{show only notes} %Notes Output

\setbeamertemplate{frametitle}{\vspace{.5em}\bfseries\insertframetitle}
\newcommand{\heading}[1]{\noindent \textbf{#1}\\ \vspace{1em}}

\usepackage{bbding,color,multirow,times,ccaption,tabularx,graphicx,verbatim,booktabs,fixltx2e}
\usepackage{colortbl} %Table overlays
\usepackage[english]{babel}
\usepackage[latin1]{inputenc}
\usepackage[T1]{fontenc}

%\author[]{Thomas J. Leeper}
\institute[]{
  \inst{}%
  Department of Political Science and Government\\Aarhus University
}


\title{Sampling Techniques and Questionnaire Design}

\date[]{September 29, 2014}

\begin{document}

\frame{\titlepage}

\frame{\tableofcontents}

\section{Stratified Sampling: An Example}
\frame{\tableofcontents[currentsection]}

\frame{
	\frametitle{Example Setup}
	\begin{itemize}\itemsep2em
		\item Interested in individual-level rate of crime victimization in Denmark
		\item We think rates differ among native-born and immigrant populations
		\item Assume immigrants make up 12\% of population
		\item Compare uncertainty from different designs ($n=1000$)
	\end{itemize}
}

\frame{
	\frametitle{SRS}
	\begin{itemize}\itemsep1em
		\item Assume equal rates across groups ($p=0.10$)
		\item Overall estimate is just $\frac{Victims}{n}$
		\item $SE(p) = \sqrt{\frac{p(1-p)}{n-1}}$
		\item $SE(p) = \sqrt{\frac{0.09}{999}} = 0.0095$
		\item<2-> SEs for subgroups (native-born and immigrants)?
		\item<3-> What happens if we don't get any immigrants in our sample?
	\end{itemize}
}


\frame{
	\frametitle{Proportionate Allocation I}
	\begin{itemize}\itemsep0.75em
		\item Assume equal rates across groups
		\item Sample 880 native-born and 120 immigrant individuals
		\item $SE(p) = \sqrt{Var(p)}$, where
		\begin{itemize}\itemsep0.75em
			\item $Var(p) = \sum_{h=1}^{H}(\frac{N_h}{N})^2 \frac{p_h(1-p_h)}{n_h - 1}$
			\item $Var(p) = (\frac{0.09}{879})(.88^2) + (\frac{0.09}{119})(.12^2)$
			\item $SE(p) = 0.0095$
		\end{itemize}
		\item Design effect: $d^2 = \frac{0.0095^2}{0.0095^2} = 1$
	\end{itemize}
}

\frame{
	\frametitle{Proportionate Allocation I}
	\begin{itemize}\itemsep0.75em
		\item Note that in this design we get different levels of uncertainty for subgroups
		\item $SE(p_{native}) = \sqrt{\frac{p(1-p)}{879}} = \sqrt{\frac{0.09}{879}} = 0.010$
		\item $SE(p_{imm}) = \sqrt{\frac{p(1-p)}{119}} = \sqrt{\frac{0.09}{119}} = 0.028$
	\end{itemize}
}



\frame{
	\frametitle{Proportionate Allocation IIa}
	\begin{itemize}\itemsep2em
		\item Assume different rates across groups (immigrants higher risk)
		\item $p_{native}=0.1$ and $p_{imm}=0.3$ (thus $p_{pop} = 0.124$)
		\item $Var(p) = \sum_{h=1}^{H}(\frac{N_h}{N})^2 \frac{p_h(1-p_h)}{n_h - 1}$
		\item $Var(p) = (\frac{0.09}{879})(.88^2) + \frac{0.21}{119})(.12^2))$
		\item $SE(p) = 0.01022$
	\end{itemize}
}

\frame{
	\frametitle{Proportionate Allocation IIa}
	\begin{itemize}\itemsep2em
		\item $SE(p) = 0.01022$
		\item Compare to SRS:
		\begin{itemize}
			\item $SE(p) = \sqrt{\frac{0.124(1-0.124)}{n-1}} = 0.0104$
		\end{itemize}
		\item Design effect: $d^2 = \frac{0.01022^2}{0.0104^2} = 0.9657$
		\item $n_{effective} = \frac{n}{sqrt(d^2)} = 1017$
	\end{itemize}
}

\frame{
	\frametitle{Proportionate Allocation IIa}
	\begin{itemize}\itemsep2em
		\item Subgroup variances are still different
		\item $SE(p_{native}) = \sqrt{\frac{p(1-p)}{879}} = \sqrt{\frac{.09}{879}} = 0.010$
		\item $SE(p_{imm}) = \sqrt{\frac{p(1-p)}{119}} = sqrt{\frac{.21}{119}} = 0.040$
	\end{itemize}
}


\frame{
	\frametitle{Proportionate Allocation IIb}
	\begin{itemize}\itemsep2em
		\item Assume different rates across groups (immigrants lower risk)
		\item $p_{native}=0.3$ and $p_{imm}=0.1$ (thus $p_{pop} = 0.276$)
		\item $Var(p) = \sum_{h=1}^{H}(\frac{N_h}{N})^2 \frac{p_h(1-p_h)}{n_h - 1}$
		\item $Var(p) = (\frac{0.21}{879})(.88^2) + \frac{0.09}{119})(.12^2))$
		\item $SE(p) = 0.014$
	\end{itemize}
}

\frame{
	\frametitle{Proportionate Allocation IIb}
	\begin{itemize}\itemsep2em
		\item $SE(p) = 0.014$
		\item Compare to SRS:
		\begin{itemize}
			\item $SE(p) = \sqrt{\frac{0.276(1-0.276)}{n-1}} = 0.0141$
		\end{itemize}
		\item Design effect: $d^2 = \frac{0.014^2}{0.0141^2} = 0.9859$
		\item $n_{effective} = \frac{n}{sqrt(d^2)} = 1007$
	\end{itemize}
}

\frame{
	\frametitle{Proportionate Allocation IIa}
	\begin{itemize}\itemsep2em
		\item Subgroup variances are still different
		\item $SE(p_{native}) = \sqrt{\frac{p(1-p)}{879}} = \sqrt{\frac{.21}{879}} = 0.0155$
		\item $SE(p_{imm}) = \sqrt{\frac{p(1-p)}{119}} = sqrt{\frac{.09}{119}} = 0.0275$
	\end{itemize}
}



\frame{
	\frametitle{Proportionate Allocation IIc}
	\begin{itemize}\itemsep2em
		\item Look at same design, but a different survey variable (household size)
		\item Assume: $\bar{y}_{native}=4$ and $\bar{Y}_imm=6$ (thus $\bar{Y}_{pop} = 4.24$)
		\item Assume: $Var(Y_{native}) = 1$ and $Var(Y_imm) = 3$ and $Var(Y_{pop}) = 4$
		\item $Var(\bar{y}) = \sum_{h=1}^{H}(\frac{N_h}{N})^2 \frac{s_h^2}{n_h}$
		\item $SE(\bar{y}) = \sqrt{\frac{1^2}{880}(.88^2) + \frac{3^2}{120}(.12^2)} = 0.0443$
	\end{itemize}
}

\frame{
	\frametitle{Proportionate Allocation IIc}
	\begin{itemize}\itemsep2em
		\item $SE(\bar{y}) = 0.0443$
		\item Compare to SRS:
		\begin{itemize}
			\item $SE(\bar{y}) = \sqrt{\frac{s^2}{n}} = \sqrt{4/1000} = 0.0632$
		\end{itemize}
		\item Design effect: $d^2 = \frac{0.0443^2}{0.0632^2} = 0.491$
		\item $n_{effective} = \frac{n}{sqrt(d^2)} = 1427$
		\item<2-> Why is $d^2$ so much larger here?
	\end{itemize}
}



\frame{
	\frametitle{Disproportionate Allocation I}
	\begin{itemize}\itemsep1em
		\item Previous designs obtained different precision for subgroups
		\item Design to obtain stratum-specific precision (e.g., $SE(p_h) = 0.02$)
		\item $n_h = \frac{p(1-p)}{v(p)} = \frac{p(1-p)}{SE^2}$
		\item $n_{native} = \frac{0.09}{0.02^2} = 225$
		\item $n_{imm} = \frac{0.21}{0.02^2} = 525$
		\item $n_{total} = 225 + 525 = 750$
	\end{itemize}
}


\frame{
	\frametitle{Disproportionate Allocation II}
	\begin{itemize}\itemsep2em
		\item Neyman optimal allocation
		\item How does this work?
		\begin{itemize}
			\item Allocate cases to strata based on within-strata variance
			\item Only works for one variable at a time
			\item Need to know within-strata variance
		\end{itemize}
	\end{itemize}
}

\frame{
	\frametitle{Disproportionate Allocation II}
	\begin{itemize}\itemsep1em
		\item Assume big difference in victimization
		\item $p_{native}=0.01$ and $p_{imm}=0.50$  (thus $p_{pop} = 0.0688$)
		\item Allocate according to: $n_h = n \frac{W_h S_h}{\sum_{h=1}^{H} W_h S_h}$
		\item $\sum_{h=1}^{H} W_h S_h = (0.88 * 0.0099) + (0.12 * 0.25) = 0.0387$
		\item $n_{native} = 1000 \frac{0.0087}{0.0387} = 225$
		\item $n_{imm} = 1000 \frac{0.03}{0.0387} = 775$
	\end{itemize}
}

\frame{
	\frametitle{Disproportionate Allocation II}
	\begin{itemize}\itemsep2em
		\item $SE(p_{native}) = \sqrt{\frac{p(1-p)}{225}} = \sqrt{\frac{0.0099}{225}} = 0.00663$
		\item $SE(p_{imm}) = \sqrt{\frac{p(1-p)}{775}} = \sqrt{\frac{.25}{775}} = 0.01796$
		\item $Var(p) = \sum_{h=1}^{H}(\frac{N_h}{N})^2 \frac{p_h(1-p_h)}{n_h - 1}$
		\item $Var(p) = (\frac{0.0099}{225})(.88^2) + (\frac{0.25}{775})(.12^2)$
		\item $SE(p) = 0.00622$
	\end{itemize}
}

\frame{
	\frametitle{Disproportionate Allocation II}
	\begin{itemize}\itemsep2em
		\item $SE(p) = 0.00622$
		\item Compare to SRS:
		\begin{itemize}
			\item $SE(p) = \sqrt{\frac{0.0688(1-0.0688)}{n-1}} = 0.008$
		\end{itemize}
		\item Design effect: $d^2 = \frac{0.00622^2}{0.008^2} = 0.6045$
		\item $n_{effective} = \frac{n}{sqrt(d^2)} = 1286$
	\end{itemize}
}

\frame{
	\frametitle{Final Considerations}
	\begin{itemize}\itemsep2em
		\item Reductions in uncertainty come from creating homogeneous groups
		\item Estimates of design effects are variable-specific
		\item Sampling variance calculations do not factor in time, costs, or feasibility
	\end{itemize}
}


\frame{\frametitle{Questions about stratified sampling?}}

\subsection{Cluster Sampling}
\frame{\tableofcontents[currentsubsection]}

\frame{
	\frametitle{Cluster Sampling}
	\begin{itemize}\itemsep2em
		\item What is it?
		\item Why do we do?
		\item<2-> Most useful when:
		\begin{enumerate}
			\item Population has a clustered structure
			\item Unit-level sampling is expensive or not feasible
			\item Clusters are similar
		\end{enumerate}
	\end{itemize}
}

\frame{
	\frametitle{Cluster Sampling}
	\begin{itemize}\itemsep2em
		\item Advantages
		\begin{itemize}
			\item<2-> Cost savings!
			\item<2-> Capitalize on clustered structure
		\end{itemize}
		\item<3-> Disadvantages
		\begin{itemize}
			\item<4-> Units tend to cluster for complex reasons (self-selection)
			\item<4-> Major increase in uncertainty if clusters differ from each other
			\item<4-> Complex to design (and possibly to administer)
			\item<4-> Analysis is much more complex than SRS or stratified sample
		\end{itemize}
	\end{itemize}
}


\frame{
	\frametitle{Example: Burnham et al.}
	\begin{itemize}\itemsep2em
		\item What is the research question?
		\item<2-> What are the population and unit of analysis?
		\item<3-> What is the sampling strategy? Why?
		\item<4-> What do they find?
	\end{itemize}
}

\frame{\frametitle{Questions about cluster sampling?}}

\section{Complex Survey Designs}
\frame{\tableofcontents[currentsection]}

\frame{
	\frametitle{Complex Survey Designs}
	\begin{itemize}\itemsep2em
		\item Often stratification and clustering are used together
		\item The choice of design must:
			\begin{itemize}
				\item Be worth increased cost
				\item Improve efficiency
			\end{itemize}
		\item Design effects
		\item Weights
	\end{itemize}
}



\section{Questionnaire Design}
\frame{\tableofcontents[currentsection]}





\section{Preview of Next Week}
\frame{\tableofcontents[currentsection]}

\frame{
	\frametitle{Assignment for next week}
	\begin{itemize}\itemsep1em
		\item What constructs/concepts do you intend to measure in your survey?
		\item How do you plan to measure these?
		\item How have these constructs been operationalized in other research?
	\end{itemize}
}

\frame{
	\frametitle{Next week's agenda}
	\begin{itemize}\itemsep1em
		\item Continue questionnaire design
		\item Measuring sensitive information
		\item Measuring knowledge
		\item Reference periods
	\end{itemize}
}


\appendix
\frame{}

\end{document}
