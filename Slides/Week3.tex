\documentclass[compress, 12pt]{beamer} %Makes presentation

%\documentclass[handout]{beamer} %Makes Handouts
\usetheme{Singapore} %Gray with fade at top
\useoutertheme[subsection=false]{miniframes} %Supppress subsection in header
\useinnertheme{rectangles} %Itemize/Enumerate boxes
\usecolortheme{seagull} %Color theme
\usecolortheme{rose} %Inner color theme

\definecolor{light-gray}{gray}{0.75}
\definecolor{dark-gray}{gray}{0.55}
\setbeamercolor{item}{fg=light-gray}
\setbeamercolor{enumerate item}{fg=dark-gray}

\setbeamertemplate{navigation symbols}{}
\setbeamertemplate{mini frames}[default]
\setbeamercovered{dynamics}
\setbeamerfont*{title}{size=\Large,series=\bfseries}

%\setbeameroption{notes on second screen} %Dual-Screen Notes
%\setbeameroption{show only notes} %Notes Output

\setbeamertemplate{frametitle}{\vspace{.5em}\bfseries\insertframetitle}
\newcommand{\heading}[1]{\noindent \textbf{#1}\\ \vspace{1em}}

\usepackage{bbding,color,multirow,times,ccaption,tabularx,graphicx,verbatim,booktabs,fixltx2e}
\usepackage{colortbl} %Table overlays
\usepackage[english]{babel}
\usepackage[latin1]{inputenc}
\usepackage[T1]{fontenc}

%\author[]{Thomas J. Leeper}
\institute[]{
  \inst{}%
  Department of Political Science and Government\\Aarhus University
}


\title{Sampling Techniques}

\date[]{September 22, 2014}

\begin{document}

\frame{\titlepage}

\frame{\tableofcontents}

\section{Assignment}
\frame{\tableofcontents[currentsection]}

\frame{
	\frametitle{Assignment for this week}
	\begin{enumerate}\itemsep1em
		\item Form groups of 3 (or so)
		\item Discuss the sampling plans for the surveys you identified online
		\item Select one of those from your group to present to the class
		\item Think about: coverage, representativeness, sample size
	\end{enumerate}
}


\section{Review of Last Week}
\frame{\tableofcontents[currentsection]}

\frame<1-4>[label=lastweek]{
	\frametitle{Ideas from last week}
	\begin{enumerate}\itemsep1em
		\item<1-> What is a population?
		\item<2-> What is a sampling frame?
		\item<3-> What is a sample?
		\item<4-> How do we construct a sampling frame? % then do sampling frame activity
		\item<5-> What is the process of determining necessary sample size for a study?
	\end{enumerate}
}

\frame{
	\frametitle{Activity!}
	\begin{itemize}\itemsep2em
		\item Work in pairs
		\item Pick one of the two populations
		\item Develop two sampling frames/sampling strategies for a population
		\item Share with class and discuss
	\end{itemize}
}


\againframe<4-5>{lastweek}

\frame{\frametitle{Questions about sampling strategies?}}

\section{New Material to Cover}
\frame{\tableofcontents[currentsection]}

\subsection{Total Survey Error}
\frame{\tableofcontents[currentsubsection]}

\frame{
	\frametitle{Total Survey Error}
	\begin{itemize}\itemsep2em
	\item What sources of survey error have we discussed so far?
	\item<2-> Now we also think about sampling error
	\end{itemize}
}


\frame{
	\frametitle{Sampling Error}
	\begin{itemize}\itemsep2em
		\item Definition?
		\item<2-> Unavoidable!
		\item<3-> Sources of sampling error:
			\begin{itemize}
				\item Sampling
				\item Sample size
				\item Unequal probabilities of selection
				\item Non-Stratification
				\item Cluster sampling
			\end{itemize}
	\end{itemize}
}


\subsection{Readings}
\frame{\tableofcontents[currentsubsection]}

\frame{
	\frametitle{Readings for this week}
	\begin{enumerate}\itemsep1em
		\item<1-> Walter and Enticott
		\item<2-> Reinisch et al.
		\item<3-> AAPOR Report
	\end{enumerate}
}




\subsection{Online Panels}
\frame{\tableofcontents[currentsubsection]}

\frame{
	\frametitle{Online panels/Non-Probability Surveys}
	\begin{itemize}
	\item What are the major issues raised in the AAPOR Report?
	\item<2-> How are online panelists recruited?
	\item<3-> How good is the coverage for an online panel? How would we evaluate it?
	\item<4-> How are panelists recruited into studies?
	\item<5-> Does stratified sampling of panelists solve concerns about representativeness?
	\item<6-> How do we assess response rates for an online panel?
	\item<7-> How long should someone be eligible to be in an online panel?
	\end{itemize}
}

\frame{
	\frametitle{Purposive and Quota Sampling}
	\begin{itemize}\itemsep1em
	\item What is purposive sampling?
	\item What is quota sampling?
	\item What concerns do we have about these methods?
	\item When are they appropriate?
	\end{itemize}
}

\frame{\frametitle{Questions about non-probability sampling?}}

\subsection{Stratified Sampling}
\frame{\tableofcontents[currentsubsection]}

\frame{
	\frametitle{Simple Random Sampling (SRS)}
	\begin{itemize}\itemsep2em
		\item Advantages
			\begin{itemize}
				\item Simplicity of sampling
				\item Simplicity of analysis
			\end{itemize}
		\item Disadvantages
			\begin{itemize}
				\item Need complete sampling frame
				\item Possibly expensive
			\end{itemize}
	\end{itemize}
}

\frame{
	\frametitle{Stratified Sampling}
	\begin{itemize}\itemsep2em
		\item What is it?
		\item Why do we do?
		\item<2-> Most useful when subpopulations are:
		    \begin{enumerate}
		    \item identifiable in advance
		    \item differ from one another
		    \item have low within-stratum variance
		    \end{enumerate}
	\end{itemize}
}

\frame{
	\frametitle{Stratified Sampling}
	\begin{itemize}\itemsep2em
		\item Advantages
			\begin{itemize}
				\item<2-> Avoid certain kinds of sampling errors
				\item<2-> Representative samples of subpopulations
				\item<2-> Often, lower variances (greater precision of estimates)
			\end{itemize}
		\item<3-> Disadvantages
			\begin{itemize}
				\item<4-> Need complete sampling frame
				\item<4-> Possibly (more) expensive
				\item<4-> No advantage if strata are similar
				\item<4-> Analysis is more potentially more complex than SRS
			\end{itemize}
	\end{itemize}
}

\frame{
	\frametitle{Outline of Process}
	\begin{enumerate}\itemsep1em
		\item Identify our population
		\item Construct a sampling frame
		\item Identify variables we already have that are related to our survey variables of interest
		\item Stratify or subset or sampling frame based on these characteristics
		\item Collect an SRS (of some size) within each stratum
		\item Aggregate our results
	\end{enumerate}
}

\frame{
	\frametitle{Estimates from a stratified sample}
	\begin{itemize}\itemsep2em
		\item Within-strata estimates are calculated just like an SRS
		\item Within-strata variances are calculated just like an SRS
		\vspace{1em}
		\item Sample-level estimates are weighted averages of stratum-specific estimates
		\item Sample-level variances are weighted averages of strataum-specific variances
	\end{itemize}
}


\frame{
	\frametitle{Design effect}
	\begin{itemize}\itemsep2em
        \item What is it?
		\item<2-> Ratio of variances in a design against a same-sized SRS
		\item<3-> $d^2 = \frac{Var_{stratified}(y)}{Var_{SRS}(y)}$
		\item<4-> Possible to convert design effect into an \textit{effective sample size}:
		\item<4-> $n_{effective} = \frac{n}{d}$
	\end{itemize}
}


\frame{
	\frametitle{How many strata?}
	\begin{itemize}\itemsep2em
		\item How many strata can we have in a stratified sampling plan?
		\item<2-> As many as we want, up to the limits of sample size
	\end{itemize}
}


\frame{
	\frametitle{How do we allocate sample units to strata?}
	\begin{itemize}\itemsep2em
		\item Proportional allocation
		\item Optimal precision
		\item Allocation based on stratum-specific precision objectives
	\end{itemize}
}


\frame{\frametitle{Questions about stratified sampling?}}

\subsection{An Extended Example}
\frame{\tableofcontents[currentsubsection]}

\frame{
	\frametitle{Example Setup}
	\begin{itemize}\itemsep2em
		\item Interested in individual-level rate of crime victimization in Denmark
		\item We think rates differ among native-born and immigrant populations
		\item Assume immigrants make up 12\% of population
		\item Compare uncertainty from different designs ($n=1000$)
	\end{itemize}
}

\frame{
	\frametitle{SRS}
	\begin{itemize}\itemsep1em
		\item Assume equal rates across groups ($p=0.10$)
		\item Overall estimate is just $\frac{Victims}{n}$
		\item $SE(p) = \sqrt{\frac{p(1-p)}{n-1}}$
		\item $SE(p) = \sqrt{\frac{0.09}{999}} = 0.0095$
		\item<2-> SEs for subgroups (native-born and immigrants)?
		\item<3-> What happens if we don't get any immigrants in our sample?
	\end{itemize}
}


\frame{
	\frametitle{Proportionate Allocation I}
	\begin{itemize}\itemsep0.75em
		\item Assume equal rates across groups
		\item Sample 880 native-born and 120 immigrant individuals
		\item $SE(p) = \sqrt{Var(p)}$, where
		    \begin{itemize}\itemsep0.75em
		    \item $Var(p) = \sum_{h=1}^{H}(\frac{N_h}{N})^2 \frac{p_h(1-p_h)}{n_h - 1}$
		    \item $Var(p) = (\frac{0.09}{879})(.88^2) + (\frac{0.09}{119})(.12^2)$
		    \item $SE(p) = 0.0095$
		    \end{itemize}
		\item Design effect: $d^2 = \frac{0.0095^2}{0.0095^2} = 1$
	\end{itemize}
}

\frame{
	\frametitle{Proportionate Allocation I}
	\begin{itemize}\itemsep0.75em
		\item Note that in this design we get different levels of uncertainty for subgroups
		\item $SE(p_{native}) = \sqrt{\frac{p(1-p)}{879}} = \sqrt{\frac{0.09}{879}} = 0.010$
		\item $SE(p_{imm}) = \sqrt{\frac{p(1-p)}{119}} = \sqrt{\frac{0.09}{119}} = 0.028$
	\end{itemize}
}



\frame{
	\frametitle{Proportionate Allocation IIa}
	\begin{itemize}\itemsep2em
		\item Assume different rates across groups (immigrants higher risk)
		\item $p_{native}=0.1$ and $p_{imm}=0.3$ (thus $p_{pop} = 0.124$)
		\item $Var(p) = \sum_{h=1}^{H}(\frac{N_h}{N})^2 \frac{p_h(1-p_h)}{n_h - 1}$
		\item $Var(p) = (\frac{0.09}{879})(.88^2) + \frac{0.21}{119})(.12^2))$
		\item $SE(p) = 0.01022$
	\end{itemize}
}

\frame{
	\frametitle{Proportionate Allocation IIa}
	\begin{itemize}\itemsep2em
        \item $SE(p) = 0.01022$
		\item Compare to SRS:
		    \begin{itemize}
		        \item $SE(p) = \sqrt{\frac{0.124(1-0.124)}{n-1}} = 0.0104$
		    \end{itemize}
		\item Design effect: $d^2 = \frac{0.01022^2}{0.0104^2} = 0.9657$
		\item $n_{effective} = \frac{n}{sqrt(d^2)} = 1017$
	\end{itemize}
}

\frame{
	\frametitle{Proportionate Allocation IIa}
	\begin{itemize}\itemsep2em
        \item Subgroup variances are still different
        \item $SE(p_{native}) = \sqrt{\frac{p(1-p)}{879}} = \sqrt{\frac{.09}{879}} = 0.010$
        \item $SE(p_{imm}) = \sqrt{\frac{p(1-p)}{119}} = sqrt{\frac{.21}{119}} = 0.040$
	\end{itemize}
}


\frame{
	\frametitle{Proportionate Allocation IIb}
	\begin{itemize}\itemsep2em
		\item Assume different rates across groups (immigrants lower risk)
		\item $p_{native}=0.3$ and $p_{imm}=0.1$ (thus $p_{pop} = 0.276$)
		\item $Var(p) = \sum_{h=1}^{H}(\frac{N_h}{N})^2 \frac{p_h(1-p_h)}{n_h - 1}$
		\item $Var(p) = (\frac{0.21}{879})(.88^2) + \frac{0.09}{119})(.12^2))$
		\item $SE(p) = 0.014$
	\end{itemize}
}

\frame{
	\frametitle{Proportionate Allocation IIb}
	\begin{itemize}\itemsep2em
		\item $SE(p) = 0.014$
		\item Compare to SRS:
		    \begin{itemize}
		        \item $SE(p) = \sqrt{\frac{0.276(1-0.276)}{n-1}} = 0.0141$
		    \end{itemize}
		\item Design effect: $d^2 = \frac{0.014^2}{0.0141^2} = 0.9859$
		\item $n_{effective} = \frac{n}{sqrt(d^2)} = 1007$
	\end{itemize}
}

\frame{
	\frametitle{Proportionate Allocation IIa}
	\begin{itemize}\itemsep2em
        \item Subgroup variances are still different
        \item $SE(p_{native}) = \sqrt{\frac{p(1-p)}{879}} = \sqrt{\frac{.21}{879}} = 0.0155$
        \item $SE(p_{imm}) = \sqrt{\frac{p(1-p)}{119}} = sqrt{\frac{.09}{119}} = 0.0275$
	\end{itemize}
}



\frame{
	\frametitle{Proportionate Allocation IIc}
	\begin{itemize}\itemsep2em
		\item Look at same design, but a different survey variable (household size)
		\item Assume: $\bar{y}_{native}=4$ and $\bar{Y}_imm=6$ (thus $\bar{Y}_{pop} = 4.24$)
		\item Assume: $Var(Y_{native}) = 1$ and $Var(Y_imm) = 3$ and $Var(Y_{pop}) = 4$
		\item $Var(\bar{y}) = \sum_{h=1}^{H}(\frac{N_h}{N})^2 \frac{s_h^2}{n_h}$
		\item $SE(\bar{y}) = \sqrt{\frac{1^2}{880}(.88^2) + \frac{3^2}{120}(.12^2)} = 0.0443$
	\end{itemize}
}

\frame{
	\frametitle{Proportionate Allocation IIc}
	\begin{itemize}\itemsep2em
		\item $SE(\bar{y}) = 0.0443$
		\item Compare to SRS:
		    \begin{itemize}
		        \item $SE(\bar{y}) = \sqrt{\frac{s^2}{n}} = \sqrt{4/1000} = 0.0632$
		    \end{itemize}
		\item Design effect: $d^2 = \frac{0.0443^2}{0.0632^2} = 0.491$
		\item $n_{effective} = \frac{n}{sqrt(d^2)} = 1427$
		\item<2-> Why is $d^2$ so much larger here?
	\end{itemize}
}



\frame{
	\frametitle{Disproportionate Allocation I}
	\begin{itemize}\itemsep1em
        \item Previous designs obtained different precision for subgroups
		\item Design to obtain stratum-specific precision (e.g., $SE(p_h) = 0.02$)
		\item $n_h = \frac{p(1-p)}{v(p)} = \frac{p(1-p)}{SE^2}$
		\item $n_{native} = \frac{0.09}{0.02^2} = 225$
		\item $n_{imm} = \frac{0.21}{0.02^2} = 525$
		\item $n_{total} = 225 + 525 = 750$
	\end{itemize}
}


\frame{
	\frametitle{Disproportionate Allocation II}
	\begin{itemize}\itemsep2em
		\item Neyman optimal allocation
		\item How does this work?
		    \begin{itemize}
		        \item Allocate cases to strata based on within-strata variance
		        \item Only works for one variable at a time
		        \item Need to know within-strata variance
		    \end{itemize}
	\end{itemize}
}

\frame{
	\frametitle{Disproportionate Allocation II}
	\begin{itemize}\itemsep1em
		\item Assume big difference in victimization
		\item $p_{native}=0.01$ and $p_{imm}=0.50$  (thus $p_{pop} = 0.0688$)
		\item Allocate according to: $n_h = n \frac{W_h S_h}{\sum_{h=1}^{H} W_h S_h}$
		\item $\sum_{h=1}^{H} W_h S_h = (0.88 * 0.0099) + (0.12 * 0.25) = 0.0387$
		\item $n_{native} = 1000 \frac{0.0087}{0.0387} = 225$
		\item $n_{imm} = 1000 \frac{0.03}{0.0387} = 775$
	\end{itemize}
}

\frame{
	\frametitle{Disproportionate Allocation II}
	\begin{itemize}\itemsep2em
		\item $SE(p_{native}) = \sqrt{\frac{p(1-p)}{225}} = \sqrt{\frac{0.0099}{225}} = 0.00663$
		\item $SE(p_{imm}) = \sqrt{\frac{p(1-p)}{775}} = \sqrt{\frac{.25}{775}} = 0.01796$
		\item $Var(p) = \sum_{h=1}^{H}(\frac{N_h}{N})^2 \frac{p_h(1-p_h)}{n_h - 1}$
		\item $Var(p) = (\frac{0.0099}{225})(.88^2) + (\frac{0.25}{775})(.12^2)$
		\item $SE(p) = 0.00622$
	\end{itemize}
}

\frame{
	\frametitle{Disproportionate Allocation II}
	\begin{itemize}\itemsep2em
		\item $SE(p) = 0.00622$
		\item Compare to SRS:
		    \begin{itemize}
		        \item $SE(p) = \sqrt{\frac{0.0688(1-0.0688)}{n-1}} = 0.008$
		    \end{itemize}
        \item Design effect: $d^2 = \frac{0.00622^2}{0.008^2} = 0.6045$
        \item $n_{effective} = \frac{n}{sqrt(d^2)} = 1286$
	\end{itemize}
}

\frame{
	\frametitle{Final Considerations}
	\begin{itemize}\itemsep2em
		\item Reductions in uncertainty come from creating homogeneous groups
		\item Estimates of design effects are variable-specific
		\item Sampling variance calculations do not factor in time, costs, or feasibility
	\end{itemize}
}


\frame{\frametitle{Questions about stratified sampling?}}

\subsection{Cluster Sampling}
\frame{\tableofcontents[currentsubsection]}

\frame{
	\frametitle{Cluster Sampling}
	\begin{itemize}\itemsep2em
		\item What is it?
		\item Why do we do?
		\item<2-> Most useful when:
		    \begin{enumerate}
		    \item Population has a clustered structure
		    \item Unit-level sampling is expensive or not feasible
            \item Clusters are similar
		    \end{enumerate}
	\end{itemize}
}

\frame{
	\frametitle{Cluster Sampling}
	\begin{itemize}\itemsep2em
		\item Advantages
			\begin{itemize}
				\item<2-> Cost savings!
				\item<2-> Capitalize on clustered structure
			\end{itemize}
		\item<3-> Disadvantages
			\begin{itemize}
				\item<4-> Units tend to cluster for complex reasons (self-selection)
				\item<4-> Major increase in uncertainty if clusters differ from each other
				\item<4-> Complex to design (and possibly to administer)
				\item<4-> Analysis is much more complex than SRS or stratified sample
			\end{itemize}
	\end{itemize}
}


\frame{
	\frametitle{Example: Burnham et al.}
	\begin{itemize}\itemsep2em
		\item What is the research question?
		\item<2-> What are the population and unit of analysis?
		\item<3-> What is the sampling strategy? Why?
		\item<4-> What do they find?
	\end{itemize}
}

\frame{\frametitle{Questions about cluster sampling?}}

\section{Preview of Next Week}
\frame{\tableofcontents[currentsection]}

\frame{
	\frametitle{Assignment for next week: Task}
	\begin{itemize}\itemsep1em
		\item What is your research topic/question?
		\item What is your population?
		\item What is your sampling frame? How does it over-cover or under-cover your population?
		\item How do you plan to sample?
		\item How big of a sample do you need?
	\end{itemize}
}

\frame{
	\frametitle{Assignment for next week: Procedure}
	\begin{itemize}\itemsep1em
		\item Present it in-class next week
		\item Email me your assignment (by Saturday night)
		\item Meet with me tomorrow or Wednesday
	\end{itemize}
}

\frame{
	\frametitle{Next week's agenda}
	\begin{itemize}\itemsep1em
		\item Cluster sampling
		\item Concept definition and operationalization
		\item Opinion questions and factual questions
		\item Practice developing questions
	\end{itemize}
}


\appendix
\frame{}

\end{document}
