\documentclass[compress, 12pt]{beamer} %Makes presentation

%\documentclass[handout]{beamer} %Makes Handouts
\usetheme{Singapore} %Gray with fade at top
\useoutertheme[subsection=false]{miniframes} %Supppress subsection in header
\useinnertheme{rectangles} %Itemize/Enumerate boxes
\usecolortheme{seagull} %Color theme
\usecolortheme{rose} %Inner color theme

\definecolor{light-gray}{gray}{0.75}
\definecolor{dark-gray}{gray}{0.55}
\setbeamercolor{item}{fg=light-gray}
\setbeamercolor{enumerate item}{fg=dark-gray}

\setbeamertemplate{navigation symbols}{}
\setbeamertemplate{mini frames}[default]
\setbeamercovered{dynamics}
\setbeamerfont*{title}{size=\Large,series=\bfseries}

%\setbeameroption{notes on second screen} %Dual-Screen Notes
%\setbeameroption{show only notes} %Notes Output

\setbeamertemplate{frametitle}{\vspace{.5em}\bfseries\insertframetitle}
\newcommand{\heading}[1]{\noindent \textbf{#1}\\ \vspace{1em}}

\usepackage{bbding,color,multirow,times,ccaption,tabularx,graphicx,verbatim,booktabs,fixltx2e}
\usepackage{colortbl} %Table overlays
\usepackage[english]{babel}
\usepackage[latin1]{inputenc}
\usepackage[T1]{fontenc}

%\author[]{Thomas J. Leeper}
\institute[]{
  \inst{}%
  Department of Political Science and Government\\Aarhus University
}

\usepackage{alltt}

\title{Data Management}

\date[]{November 24, 2014}

\begin{document}

\frame{\titlepage}

\frame{
	\frametitle{Data Management}
	\begin{itemize}\itemsep0.5em
		\item Weighting
		\item Handling missing data
		\begin{itemize}
			\item Categorizing missing data types
			\item Imputation
		\end{itemize}
		\item Summary measures
		\begin{itemize}
			\item Scale construction
			\item Combining question branches
		\end{itemize}
		\item Coding and editing
		\begin{itemize}
			\item Open-ended questions
			\item Marking problematic data
		\end{itemize}
		\item Data preparation
		\begin{itemize}
			\item Codebook creation
			\item File formats
			\item Archiving, access, and rights
		\end{itemize}
	\end{itemize}
}



\frame{\tableofcontents}

\section{Weighting}
\frame{\tableofcontents[currentsection]}


\frame{
	\frametitle{Goal of Survey Research}
	\begin{itemize}\itemsep2em
		\item The goal of survey research is to estimate population-level quantities (e.g., means, proportions, totals)
		\item Samples estimate those quantities with uncertainty (sampling error)
		\item Sample estimates are unbiased if they match population quantities
	\end{itemize}
}

\frame{
	\frametitle{Realities of Survey Research}
	\begin{itemize}\itemsep1em
		\item Sample may not match population for a variety of reasons:
			\begin{itemize}
				\item Due to constraints on design
				\item Due to sampling frame coverage
				\item Due to intentional over/under-sampling
				\item Due to nonresponse
				\item Due to sampling error
			\end{itemize}
		\item<2-> Weights can be used to ``correct'' a sample
		\item<3-> Weighting is never perfect
			\begin{itemize}
				\item Limited to work with observed variables
				\item Rarely have good knowledge of coverage, nonresponse, or sampling error
				\item Weighting can increase sampling variance
			\end{itemize}
	\end{itemize}
}

\frame{
	\frametitle{Three Kinds of Weights}
	\begin{itemize}\itemsep2em
		\item Design Weights
		\item Nonresponse Weights
		\item Post-Stratification Weights
	\end{itemize}
}

\frame{
	\frametitle{Design Weights}
	\begin{itemize}\itemsep2em
		\item Address design-related unequal probability of selection into a sample
		\item Applied to \textit{complex survey designs}:
			\begin{itemize}
				\item Disproportionate allocation stratified sampling
				\item Oversampling of subpopulations
				\item Cluster sampling
				\item Combinations thereof
			\end{itemize}
	\end{itemize}
}

\frame{
	\frametitle{Design Weights: Simple Random Sampling}
	\begin{itemize}\itemsep1em
		\item Imagine sampling frame of 100,000 units
		\item Sample size will be 1,000 
		\item What is the probability that a unit in the sampling frame is included in the sample?
		\item<2-> $p = \frac{1000}{100,000} = .01$
		\item<3-> Design weight for all units is $w = 1/p = 100$
		\item<3-> SRS is \textit{self-weighting}
	\end{itemize}
}

\frame{
	\frametitle{Design Weights: Stratified Sample}
	\begin{itemize}\itemsep1em
		\item Imagine sampling frame of 100,000 units
			\begin{itemize}
				\item 90,000 Danes \& 10,000 Immigrants
			\end{itemize}
		\item Sample size will be 1,000 (proportionate allocation)
			\begin{itemize}
				\item 900 Danes \& 100 Immigrants
			\end{itemize}
		\item What is the probability that a unit in the sampling frame is included in the sample?
			\begin{itemize}
				\item<2-> $p_{Danish} = \frac{900}{90,000} = .01$
				\item<2-> $p_{Imm} = \frac{100}{10,000} = .01$
			\end{itemize}
		\item<3-> Design weight for all units is $w = 1/p = 100$
		\item<3-> Proportionate allocation is \textit{self-weighting}
	\end{itemize}
}

\frame{
	\frametitle{Design Weights: Stratified Sample}
	\begin{itemize}\itemsep0.75em
		\item Imagine sampling frame of 100,000 units
		\begin{itemize}
			\item 90,000 Danes \& 10,000 Immigrants
		\end{itemize}
		\item Sample size will be 1,000 (disproportionate allocation)
		\begin{itemize}
			\item 500 Danes \& 500 Immigrants
		\end{itemize}
		\item What is the probability that a unit in the sampling frame is included in the sample?
		\begin{itemize}
			\item<2-> $p_{Danish} = \frac{500}{90,000} = .0056$
			\item<2-> $p_{Imm} = \frac{500}{10,000} = .05$
		\end{itemize}
		\item<3-> Design weights differ across units: 
			\begin{itemize}
				\item $w_{Danish} = 1/p_{Danish} = 178.57$
				\item $w_{Imm} = 1/p_{Imm} = 20$
			\end{itemize}
		\item<3-> Disproportionate allocation is not \textit{self-weighting}
	\end{itemize}
}

\frame{
	\frametitle{Design Weights: Cluster Sample}
	\begin{itemize}\itemsep0.75em
		\item Imagine sampling frame of 1000 units in 5 clusters of varying sizes
		\item Sample size will be 10 each from 3 clusters
		\item What is the probability that a unit in the sampling frame is included in the sample?
			\begin{itemize}
				\item $p = n_{clusters}/N_{clusters} * 1/n_{cluster} = \frac{3}{5} * 1/n_{cluster}$
			\end{itemize}
		\item<3-> Design weights differ across units: 
		\begin{itemize}
			\item Clusters are equally likely to be sampled
			\item Probability of selection within cluster varies with cluster size
		\end{itemize}
		\item<3-> Cluster sampling is rarely \textit{self-weighting}
	\end{itemize}
}


\frame{
	\frametitle{Nonresponse Weights}
	\begin{itemize}\itemsep2em
		\item Correct for nonresponse
		\item Require knowledge of nonrespondents on variables that have been measured for respondents
		\item Requires data are \textit{missing at random}
		\item Two common methods
			\begin{itemize}
				\item Weighting classes
				\item Propensity score subclassification
			\end{itemize}
	\end{itemize}
}

\frame{
	\frametitle{Nonresponse Weights: Example}
	\begin{itemize}\itemsep2em
		\item Imagine immigrants end up being less likely to respond\footnote{{\scriptsize This refers to a lower RR in this particular survey sample, not in general.}}
			\begin{itemize}
				\item $RR_{Danish} = 1.0$
				\item $RR_{Imm} = 0.8$
			\end{itemize}
		\item<2-> Using weighting classes:
			\begin{itemize}
				\item $w_{rr,Danish} = 1/1 = 1$
				\item $w_{rr,Imm} = 1/0.8 = 1.25$
			\end{itemize}
		\item<2-> Can generalize to multiple variables and strata
	\end{itemize}
}

\frame{
	\frametitle{Post-Stratification}
	\begin{itemize}\itemsep1em
		\item Correct for nonresponse, coverage errors, and sampling errors
		\item<2-> Reweight sample data to match population distributions
			\begin{itemize}
				\item Divide sample and population into strata
				\item Weight units in each stratum so that the weighted sample stratum contains the same proportion of units as the population stratum does
			\end{itemize}
		\item<3-> There are numerous other related techniques
	\end{itemize}
}

\frame{
	\frametitle{Post-Stratification: Example}
	\begin{itemize}\itemsep1em
		\item Imagine our sample ends up skewed on immigration status and gender relative to the population\\
		\vspace{1em}
		\small
		\begin{tabular}{lrrlr}
			\hline
			Group             & Pop. & Sample & Rep.                &              Weight \\ \hline
			Danish, Female    &  .45 &     .5 & \onslide<2->{Over}  & \onslide<3->{0.900} \\
			Danish, Male      &  .45 &     .4 & \onslide<2->{Under} & \onslide<4->{1.125} \\
			Immigrant, Female &  .05 &    .07 & \onslide<2->{Over}  & \onslide<5->{0.714} \\
			Immigrant, Male   &  .05 &    .03 & \onslide<2->{Under} & \onslide<6->{1.667}\\  \hline
		\end{tabular}
		\item PS weight is just $w_{ps} = N_l / n_l$
	\end{itemize}
}

\frame{
	\frametitle{Post-Stratification}
	\begin{itemize}\itemsep2em
		\item Should only be done after correcting for sampling design
		\item Strata must be large ($n>15$)
		\item Need accurate population-level stratum sizes
		\item Only useful if stratifying variables are related to key constructs of interest		
		\item<2-> This is the basis for inference in non-probability samples
		\begin{itemize}
			\item Probability samples make design-based inferences
			\item Non-probability samples post-stratify to obtain descriptive representativeness
		\end{itemize}
	\end{itemize}
}


\frame{\frametitle{Questions about weighting?}}

\section{Missing Data}
\frame{\tableofcontents[currentsection]}


\frame{
	\frametitle{Sources of Missing Data}
	\begin{itemize}\itemsep2em
		\item Unit or item nonresponse
		\item Attrition or break-off
		\item Data loss
	\end{itemize}
}

\frame{
	\frametitle{Effects of Missing Data}
	\begin{itemize}\itemsep2em
		\item<1-> Sampling variance and effective sample size
		\item<2-> Scale construction and multi-variate analysis
		\item<3-> Bias in estimates
	\end{itemize}
}

\frame{
	\frametitle{Imputation}
	\begin{itemize}\itemsep2em
		\item<1-> Definition\onslide<2->{: Systematic replacement of missing values}
		\item<3-> Why?
		\begin{itemize}
			\item Casewise deletion creates loss of information
			\item Preserve sampling variances (i.e., no loss of precision)
		\end{itemize}
		\item<4-> Considerations
		\begin{itemize}
			\item Why are data missing?
			\item How do we impute?
			\item What are the consequences of imputation?
		\end{itemize}
	\end{itemize}
}

\frame{
	\frametitle{Missing Data Assumptions}
	\begin{itemize}\itemsep2em
		\item Missing Completely At Random (MCAR)
		\item Missing At Random (MAR)
		\item Missing Not At Random (MNAR)
	\end{itemize}
}

\frame{
	\frametitle{Imputation Methods}
	\begin{itemize}\itemsep2em
		\item<1-> Single Imputation
		\begin{itemize}
			\item<2-> Mean imputation
			\item<2-> Top/bottom category imputation
			\item<2-> Random imputation
			\item<2-> Hot deck imputation
			\item<2-> Regression imputation
		\end{itemize}
		\item<3-> Multiple Imputation
		\begin{itemize}
			\item<4-> Single imputation multiple times, combining results across data sets
			\item<4-> Can apply numerous imputation methods
			\item<4-> Accounts for uncertainty due to missingness
		\end{itemize}
	\end{itemize}
}


\section{Coding and Data Preparation}
\frame{\tableofcontents[currentsection]}


\frame{
	\frametitle{Coding}
	\begin{itemize}\itemsep2em
		\item What is coding?
			\begin{itemize}
				\item Categorizing responses
				\item Assigning numeric values to categories
			\end{itemize}
		\item<2-> When in the data collection process do we code?
			\begin{itemize}
				\item In the field
				\item After data collection
			\end{itemize}
		\item<3-> How do we code?
			\begin{itemize}
				\item Create set of \textit{exhaustive}, \textit{mutually exclusive} categories
				\item Assign responses to categories
				\item Add new categories, as needed
			\end{itemize}
	\end{itemize}
}

\frame{
	\frametitle{Practice Coding}
	\begin{enumerate}\itemsep2em
		\item Code the Gordon Brown responses as:
			\begin{itemize}
				\item Correct
				\item Incorrect
				\item ``Don't know''
			\end{itemize}
		\item Code the MIP responses into issue categories
	\end{enumerate}
}


\frame{
	\frametitle{Data Editing}	
	\begin{itemize}\itemsep2em
		\item Leftover of manual data recording
		\item Software handles most data editing now
			\begin{itemize}
				\item Online survey tools (e.g., Qualtrics)
				\item CATI systems
			\end{itemize}
		\item May still have problematic data points that need to be marked or changed
			\begin{itemize}
				\item If still in field, may clarify answers with respondents
			\end{itemize}
	\end{itemize}
}



\frame{
	\frametitle{Anonymizing Data}
	\begin{itemize}\itemsep2em
		\item<1-> Why do data need to be anonymous?
			\begin{itemize}
				\item<2->Guarantees of anonymity
				\item<2->Sensitive data
			\end{itemize}
		\item<3-> When are data non-anonymous?
			\begin{itemize}
				\item<4->Identifying information
				\item<4->Statistical identifiability
			\end{itemize}
		\item<5-> How do we anonymize?
			\begin{itemize}
				\item<6->Restrict data access
				\item<6->Remove identifying variables
			\end{itemize}
	\end{itemize}
}

\frame{
	\frametitle{Data Storage, Archiving, and Sharing}
	\begin{itemize}\itemsep2em
		\item<1->In what formats can we store survey data?
			\begin{itemize}
				\item<2->Paper
				\item<2->Punchcards
				\item<2->Digitally
			\end{itemize}
		\item<3->Considerations in digital formats
			\begin{itemize}
				\item \textit{Open} versus \textit{proprietary}
				\item \textit{Human-readable} versus \textit{machine-readable}
				\item File sizes
				\item Study-level metadata
				\item Question-level metadata
			\end{itemize}
	\end{itemize}
}

\frame{
	\frametitle{Study-level Metadata}
	\begin{itemize}
		\item Title
		\item Creator/Author
		\item Sponsor
		\item Description
		\item Date of publication
		\item Dates of data collection
		\item Population, sampling frame, etc.
		\item Sampling design
		\item Sample size
		\item Recruitment details
		\item Mode
		\item Rights
	\end{itemize}
}

\frame{
	\frametitle{Question-level Metadata}
	\begin{itemize}\itemsep1em
		\item Response codes
		\item Response labels
		\item Variable labels
		\item Variable names
		\item Missing data categories
		\item Variable types
		\item Mode
	\end{itemize}
}

\frame{
	\frametitle{Question-level Metadata II}
	\begin{itemize}\itemsep1em
		\item Details of randomization or question order
		\item Exclusion criteria
		\item Source of data (if not from R)
		\item Frequencies or summary statistics
		\item Interviewer instructions
		\item Constraints on responses
	\end{itemize}
}


\frame{
	\frametitle{Example Codebook\footnote{{\tiny From: \href{http://www.europeansocialsurvey.org/docs/round6/survey/ESS6\_appendix\_a7\_e02\_0.pdf}{http://www.europeansocialsurvey.org/docs/round6/survey/ESS6\_appendix\_a7\_e02\_0.pdf}}}}
	\scriptsize
	\begin{alltt}
		Question B 10 DK\\
		Which party did you vote for in that election? (Denmark)\\
		Variable name and label: prtvtcdk Party voted for in last national election, Denmark\\
		Values and categories\\
		01 Socialdemokraterne - the Danish social democtrats\\
		02 Det Radikale Venstre - Danish Social-Liberal Party\\
		03 Det Konservative Folkeparti - Conservative\\
		04 SF Socialistisk Folkeparti - Socialist People's Party\\
		05 Dansk Folkeparti - Danish peoples party\\
		06 Kristendemokraterne - Christian democtrats\\
		07 Venstre, Danmarks Liberale Parti - Venstre\\
		08 Liberal Alliance - Liberal Alliance\\
		09 Enhedslisten - Unity List - The Red-Green Alliance\\
		10 Andet - other\\
		66 Not applicable\\
		77 Refusal\\
		88 Don't know\\
		99 No answer\\
		Filter: If code 1 at B9\\
	\end{alltt}
}

\frame{
	\frametitle{File Formats}
	\begin{itemize}\itemsep2em
		\item Go to the course website
		\item Open data files under Week 11
		\item All contain the same data
		\item What do you notice about the different files?
	\end{itemize}
}


\frame{
	\frametitle{Metadata Standards}
	\begin{itemize}
		\item Most survey data are stored in proprietary formats using codebooks constructed in arbitrary formats
			\begin{itemize}
				\item This makes it hard to work with survey data
			\end{itemize}
		\item There are common standards for metadata
			\begin{itemize}
				\item Dublin Core (DC)
				\item Data Documentation Initiative (DDI)
			\end{itemize}
	\end{itemize}
}

\frame{
	\frametitle{For Your Project}	
	\begin{itemize}\itemsep2em
		\item Discuss appropriate file format for data storage/sharing
		\item Discuss how data can be used after collection (i.e., rights)
		\item Discuss codebook creation
			\begin{itemize}
				\item When do you create a codebook
				\item What goes in your codebook
				\item Where do you record study-level metadata
			\end{itemize}
	\end{itemize}
}

\frame{\frametitle{Questions about handling survey data?}}

\section{Wrap-up}
\frame{\tableofcontents[currentsection]}

\frame{
\frametitle{Total Survey Error}
	\begin{itemize}\itemsep1em
		\item Design-Related Errors
			\begin{itemize}
				\item Coverage Error
				\item Sampling Error
				\item Nonresponse Error
				\item Adjustment Error
			\end{itemize}
		\item<2-> Measurement Errors
			\begin{itemize}
				\item Construct Validity
				\item Measurement Error and Response Biases
				\item Processing Error
			\end{itemize}
		\item<3-> Our goal: Minimize \textit{total} error (thus maximizing data quality), within the constraints of time, cost, and other resources
	\end{itemize}
}


\section{Preview of Next Time}
\frame{\tableofcontents[currentsection]}

\frame{
	\frametitle{Agenda for next two classes}
	\begin{itemize}\itemsep2em
		\item Presentations
		\item Prepare questions to get help with
		\item Email me if you want to meet (after Dec. 4)
	\end{itemize}
}

\appendix
\frame{}

\end{document}
