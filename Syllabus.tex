\documentclass[12pt,a4paper]{article}
\usepackage[top=1in, bottom=1in, left=1in, right=1in]{geometry}
\usepackage{graphicx,setspace,hyperref,amsmath,amsfonts,multirow,ccaption,mdwlist,comment}
% mini table of contents
\usepackage{minitoc}
\dosecttoc % make section toc
\setcounter{secttocdepth}{2} % subsection depth
\renewcommand{\stctitle}{} % no title
\nostcpagenumbers

% optionally include commented environments
\excludecomment{lessonplan}

\setlength{\marginparwidth}{.5in}
\usepackage{natbib}
% Two lines to create in-text full citations for a syllabus
% And comment out my other standard bibtex commands
\usepackage{bibentry}
\newcommand{\reading}[2][]{\noindent --{#1} from \bibentry{#2}.\vspace{.25em}\\}
\newcommand{\textbook}[2][]{\noindent --{#1} from Groves et al.\vspace{.25em}\\} % textbook reference
\newcommand{\seealso}{\noindent \emph{See Also:}\\}
\newcommand{\topic}[1]{\noindent \textbf{#1}\\}
\usepackage[T1]{fontenc}
\usepackage{lmodern}
\hypersetup{
    bookmarks=true,         % show bookmarks bar?
    unicode=false,          % non-Latin characters in Acrobat’s bookmarks
    pdftoolbar=true,        % show Acrobat’s toolbar?
    pdfmenubar=true,        % show Acrobat’s menu?
    pdffitwindow=false,     % window fit to page when opened
    pdfstartview={FitH},    % fits the width of the page to the window
    pdftitle={Syllabus: Using Surveys for Research and Evaluation},    % title
    pdfauthor={Thomas J. Leeper},     % author
    pdfsubject={Political Science},   % subject of the document
    pdfnewwindow=true,      % links in new window
    pdfborder={0 0 0}
}

\title{Using Surveys for Research and Evaluation}
\author{Thomas J. Leeper\\
Department of Political Science and Government\\
Aarhus University}

\begin{document}
\nobibliography*

\maketitle

\faketableofcontents

%\section{Introduction}

Almost every organization --- public or private --- conducts surveys to understand their customers, clients, constituents, research subjects, and the public at-large. Surveys are therefore a critically important methodological tool for conducting empirical political science research and for evaluating policy, performance, and impact. Nearly every political science student will, in the course of their future career, have to conduct, analyze, or interpret survey-based research.

How do we conduct surveys well so that they provide meaningful insights? How do we write survey questions so that respondents understand them and that those responses generate useful data about relevant constructs? How do we decide whom to interview and how do we interview them effectively? How do we prepare for and manage the implementation of a large survey? How do we analyze the obtained results? This course will provide answers to these questions while training students how to implement a survey data collection and analyze the results thereof. The exam will entail the design, preparation, and pilot testing of a survey on the topic of student's choice.

\section{Objectives}
The learning objectives for the course are as follows. By the end of the course, students should be able to:

\begin{enumerate}
\item Identify and analyze the usefulness of survey methods for conducting research and evaluating programs, policies, and outcomes
\item Identify sources of error and bias in surveys
\item Explain how sampling procedures enable estimation of population parameters and evaluate the implications of differences in sampling techniques
\item Demonstrate how to successfully manage large survey data collection efforts
\item Evaluate the quality of surveys in terms of sampling, interviewing procedures, and questionnaire content 
\item Apply methodological and substantive knowledge from the course to the design and implementation of an original survey
\end{enumerate}

\section{Exam and Assignments}
The exam for the course involves a home assignment, totaling 8000 words. The exam will take the form of an essay that describes a research question, hypotheses, sampling design, operationalization of constructions, implementation plan, and pilot testing of a proposed survey. A complete questionnaire for that survey, including exact question wordings of all core constructs, must be created and included as a supplemental appendix to the exam paper.

While officially due at the end of the course, each requisite portion of the exam will complete during the course in the form of weekly written assignments that progressively evaluate student learning. As such, the exam is a ``portfolio'' of these earlier assignments, which will need to be made into a single, coherent, finished product reflecting peer and instructor feedback from earlier. The exam paper should also include a concluding section (approximately one page) reflecting on the evolution of your survey plan from its initial origins at the beginning of the course to its final state, with emphasis on how what you learned in the course influenced your final survey design.

Details on the weekly assignments are listed in the course schedule. Assignments are due on the day they are described in the schedule.



%\clearpage
\section{Reading Material}
The assigned material for the course includes a textbook and empirical research articles, all of which are available online. All readings should be completed for the day they are described. \textbf{There is reading assigned on the first day.} The textbook for the course is:\\

\reading{Grovesetal2004}

\clearpage
\section{Schedule}
The general schedule for the course is as follows. Details on the readings for each week are provided on the following pages.

\secttoc

\clearpage


% possible outside people
% Kim: matching survey data to registry data
% Gitte: qualitative interviewing
% Helene: elite interviewing
% Rune: survey experiments
% Stubager: Election study
% Someone from public administration (Morten Jakobsen?): interviewing public employees



\subsection{No meeting (Week 36)}
\emph{Topic}
\vspace{1em}

\subsubsection*{Assignment Due}
None

\subsubsection*{Readings}
\textbook[Ch.1]{}
%\reading{ConverseBookSection} % maybe Jean Converse history text
\reading{Brady2000} % overview of surveys in political science
%\seealso




\clearpage
\subsection{What can surveys tell us? (Week 37)}
\emph{Topic}
% introduction to surveys: why do we do them; what do they tell us; examples
% polling
% election studies
% census measurement (registry; not all countries do)
% media usage (TNS Gallup)
% measure prevalence of something
% assess public views of something
% assess exposure
% pre- and post-event measures (evaluation)
% measure non-individual units: companies, municipalities, organizations, parties
% elite surveys: members of parliament, business leaders, etc.


\vspace{1em}
\subsubsection*{Assignment Due}
Find a survey online. This can be any survey (a national election study, a political poll, a survey of childhood health, a survey of political interest organizations, a survey of businesses, etc.). Write one page describing the survey project's research objective, the population of cases being surveyed, any important details about how the survey is designed or implemented, and the constructs measured in the survey. Briefly evaluate why the survey is interesting or important. Be prepared to share a summary if this essay in class.

\subsubsection*{Readings}
\textbook[Ch.2]{} % total survey error

\seealso




\clearpage
\subsection{Survey Designs for Research and Evaluation (Week 38)}
\emph{Topic}
\vspace{1em}
\subsubsection*{Assignment Due}

\subsubsection*{Readings}

\reading[Ch.4--5]{ShadishCookCampbell2002}

% cross-sections, rolling cross-sections, panels, rolling panels, etc.
\reading{JohnstonBrady2002}
\reading{Clinton2001} % panel attrition

% survey experiments
\reading{GainesKuklinskiQuirk2007}
\reading{DruckmanPetersonSlothuus2013} % maybe


\seealso



\clearpage
\subsection{Survey Sampling (Week 39)}
\emph{Topic}
\vspace{1em}
\subsubsection*{Assignment Due}
What do you want to know? What is your research question? In one page, describe a topic of interest to political science that you can address with a survey. It can be a question about the prevalence of something (e.g., a condition or behavior), the level of something (e.g., opinions or income), the effect of an intervention on an outcome (e.g., an outcome expected to respond to a randomized treatment or a real-world policy), or similar. Describe your topic and your research question. Then, describe what construct or constructs you need to measure in your survey in order answer your question. Speculate briefly about how you might measure those constructs in a survey.

\subsubsection*{Readings}
\textbook[Ch.3--4]{}

% sampling: unit (HH, individual, company, worker, party, municipality, farms, libraries, high school classrooms, etc.); population
% sampling frame: address, telephone (landline, mobile), registry
% sampling: SRS; cluster; stratified; convenience; online panels
\reading{Bakeretal2010}
\reading{Berinskyetal2011}
\reading{Casseseetal2013}

\seealso

\subsubsection*{In-class Activities}
\begin{itemize}
\item Identifying possible sampling frames
\item Constructing a sampling frame
\item Simple Random Sampling (SRS) from within a frame
\item Stratified and cluster sampling from a frame
\item Mean and proportion estimates, and their variances
\end{itemize}




\clearpage
\subsection{Questionnaire Design I (Week 40)}
\emph{Topic}
% constructs and operationalization


\vspace{1em}
\subsubsection*{Assignment Due}
What is your population of cases that you intend to survey? Do you plan to do a census or only interview a sample of the population? If a sample, what is the sampling frame and how are individuals selected from it? How large of a sample do you plan to collect to obtain sufficiently precise estimates of constructs? In one written page, provide answers to these questions and be prepared to discuss your plans in class.


\subsubsection*{Readings}
\reading{SchaefferPresser2003} % overview
\reading{AnsolabehereRoddenSnyder2008} % measurement error


% open and closed
\reading{SchumanPresser1979}
\reading{BrewerGross2005} % substantive example

% -- opinion questions
\reading{KrosnickJuddWittenbrink2005}
\reading{Krosnicketal2002} % ``no opinion'' options

% -- different types of questions (factual, demographic, autobiographical, opinion, behaviors, etc.)
\reading{MillerOrr2008} % DK options
\reading{NadeauNiemi1995} % knowledge questions
\reading{Prior2014} % visual political knowledge

% -- reference periods
\reading{Price} % journalism quarterly
\reading{BurtonBlair2011} % maybe
% -- response categories (differentials, agree/disagree, ratings, rankings, therms)
\reading{RevillaSarisKrosnick2013}
\reading{WilcoxSigelmanCook1989} % thermometers



\seealso



\clearpage
\subsection{Questionnaire Design II (Week 41)}
\emph{Topic}
\vspace{1em}
\subsubsection*{Assignment Due}
Given your research question

\subsubsection*{Readings}
\textbook[Ch.7]{}


% -- sensitive questions
\reading{HolbrookKrosnick2010}
\reading{Glynn2013}
\reading{TourangeauSmith1996}

\reading{TraugottKatosh1979} % vote validation
\reading{HolbrookKrosnick2013} % turnout wording


% registry data
\reading{SoenderskovDinesen2014} % in Danish, maybe
\reading{Ottosen2011} % Danish survey-registry match (child study)
\reading{DavidsenKjoellerHelwegLarsen2011} % Danish survey-registry match (DANCOS)



\seealso


\clearpage
\subsection{No class (Week 42)}

\clearpage
\subsection{Survey Mode (Week 43)}
\emph{In what mode, or format, do respondents provide answers to questions? Survey interviewing was originally entirely face-to-face, with interviewers reading questions aloud and recording answers on paper. With advances in both technology and scientific understanding of survey responding, there are now numerous modes and formats in which respondents can provide answers. What impact does mode have on responding? How does it affect quality, validity, and cost of surveys? And how does mode influence the kinds of questions that can be asked and the way that respondents engage with the survey interview?}
\vspace{1em}

\subsubsection*{Assignment Due}
Be prepared to share and briefly present a complete draft of your questionnaire. This does not need to be finalized, as you may want to add or delete questions, or change question wordings, response categories, or orderings particularly in-light of discussions about survey mode.

\subsubsection*{Readings}
\textbook[Ch.5]{}
% survey mode: ftf, telephone (landline/mobile), online; CASI; CATI
\reading{KreuterPresserTourangeau2009}
\reading{VillarCallegaroYang2013} % progress meters
\reading{Couperetal2013} % grids
\reading{MedwayFulton2012} % web response option
\reading{Smythetal2006}


\seealso



\clearpage
\subsection{Questionnaire Design III (Week 44)}
\emph{Now that you've written a full questionnaire and thought about how you'll gather answers to its questions from respondents.}
\vspace{1em}
\subsubsection*{Assignment Due}

\subsubsection*{Readings}
% -- question ordering
\reading{BishopOldendickTuchfarber1984} % political interest
\reading{TourangeauRasinski1988} % context effects
\reading[Ch.2]{SchumanPresser1996}

% web-specific considerations
% paradata; interviewer-collected data


\seealso




\clearpage
\subsection{Survey Evaluation and Pilot Testing (Week 45)}
\emph{Topic}
\vspace{1em}
\subsubsection*{Assignment Due}



\subsubsection*{Readings}
\textbook[Ch.8]{}

\reading{Presseretal2004}
\reading{PresserBlair1994}

\reading{MillerGroves1985} % official records


\seealso




\clearpage
\subsection{Interactions with Interviewers or Instruments (Week 46)}
\emph{Topic}
\vspace{1em}
\subsubsection*{Assignment Due}
In one page, describe a plan for pilot testing your survey. What techniques will you use to assess your questions, your planned survey mode, and the overall quality of your instrument? How many people do you need to pilot the survey on? Be prepared to discuss these plans in detail and revise them in response to feedback during class.

\subsubsection*{Readings}
\textbook[Ch.9, 11]{}
% response behavior
\reading{RederRitter1992} % Feeling of knowing
\reading{BishopTuchfarberOldendick1986} % fictitious issues
\reading{Davis1997} % race of interviewer
\reading{Krosnick1991} % satisficing
\reading{Prior2009b} % overreported exposure
\reading{JensenThomsen2013} % self-reported cheating
% social desirability bias


% something about survey ethics: we are inviting ourselves into peoples' lives and asking them to reveal things they might not otherwise say

\seealso



\clearpage
\subsection{Fielding (Week 47)}
\emph{Topic}
\vspace{1em}
\subsubsection*{Assignment Due}
Considering the feedback on the previous assignment and the new details you've learned about respondent behavior and the interactions between respondents and interviewers, begin implementing the pilot testing of your survey. In 1--2 written pages, report your initial findings from pilot testing, reflect on what those findings mean for your planned survey, and describe changes you will make to your survey plan based on the pilot testing.


\subsubsection*{Readings}
% recruitment
\reading{Dykemaetal2013}
% interviewing/fielding/training, reliability checks, interviewer effects, interviewer biases
\reading{Kaplowitzetal2011} % invitations
\reading{SchoberConrad1997} % flexible interviewing

% Krosnick paper on shocking behavior

% participation incentives
\reading{MartinAbreuWinters2001} % refusal conversion


% challenging areas
\reading{DirscollLidow2014} % Mogadishu, Somali paper
\reading{LyallBlairImai2013} % Afghanistan survey experiment


\seealso




\clearpage
\subsection{Nonresponse and Data Management (Week 48)}
\emph{Topic}
\vspace{1em}
\subsubsection*{Assignment Due}
% something about writing a field plan



\subsubsection*{Readings}
\textbook[Ch.6, 10]{}
% nonresponse: unit/item, adaptive survey design
\reading{CurtinPresserSinger2005}
\reading{Groves2006}
\reading{Berinsky2002}

\reading{BehrBellgardtRendtel2005}

% nonresponse rates vs. nonresponse bias

% weighting

% data management and archiving, codebooks, privacy, cleaning/missingness
% paradata


\seealso





\clearpage
\subsection{Student Presentations (Weeks 49--51, as needed)}
\emph{Topic}
\vspace{1em}
\subsubsection*{Assignment Due}
Each student will distribute a copy of their survey two days prior to class and then provide a 10--15 minute presentation of their final exam paper. In the presentation, you should present your research question, sampling plan, and details of the survey mode and questionnaire. Based on details from last week, be prepared to discuss issues of refusals, nonresponse, and dropoff/attrition.

When not presenting, provide feedback to fellow students on their planned surveys.

\subsubsection*{Readings}
None assigned. Surveys or any other materials from related to student presentations.



% load bibtext, but don't generate a bibliography
\bibliographystyle{plain}
\nobibliography{Syllabi}

\end{document}
