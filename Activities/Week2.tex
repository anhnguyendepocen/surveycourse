\documentclass[a4, 12pt]{article}
\usepackage[top=2cm, bottom=2cm, left=2cm, right=2cm]{geometry}
\usepackage{setspace}

%opening
\title{Sampling Frames and Strategies}
\author{}
\date{}

\begin{document}

\maketitle

\onehalfspacing

\section*{Instructions}

The purpose of this activity is to think about how to define a population of interest, construct and evaluate a sampling frame for that population, and conduct simple random sampling from the frame.

Each item below describes a population that we would like to sample for a research study. Develop two (2) distinct sampling frames (or other sampling strategies) that could be used to sample from this population. For each, evaluate the quality of the sampling frame in obtaining a representative sample of the target population.

In evaluating the quality of the sampling frame, consider some of the following questions: What sources of undercoverage or overcoverage are likely to occur in that frame? How severe are the problems? Are there any challenges in obtaining the information necessary to construct this sampling frame? If it is not possible to fully enumerate a sampling frame, what alternative random sampling procedure might be used to produce a representative sample of the population?


\singlespacing

\section*{Populations}

\begin{enumerate}\itemsep2em

\item All individuals who voted in the 2014 European Parliamentary elections in Denmark.

\item All visitors to ARoS art museum on September 16, 2014.

\item All members of all Danish biker gangs.

\item All high school teachers in Denmark, Norway, and Sweden.

\item All undocumented immigrants (i.e., those without a travel visa or residence permit) in the European Union.

\item All candidates for state legislative office in the United States in 2014.

\item All individuals in Denmark that commute to work via bicycle.

\item All children born in the United Kingdom to immigrant parents.

\item All individuals who visited the Aarhus Botanic Garden between April 1 and August 31, 2014.

\item All households in Denmark that live more than a 15 minute walk from a supermarket.

\item All individuals in Germany who have at any point in time paid for sex (or any sex act).

\item All victims of property crimes (e.g., damage, burglary, theft) that occurred in Denmark in 2013.

\item All individuals currently working as newspaper journalists in Denmark.

\item All international development aid organizations (NGOs) working in the countries of Sub-Saharan Africa.

\item All visits to the Jobcenter in Aarhus Municipality.\footnote{Note: Here we are interested in sampling individual visits to the Jobcenter, not individual citizens who have visited the job center. For example, we might want to know how satisfied a visitor is during each visit to the Jobcenter to assess the organization's customer service.}

\item All teenagers (ages 13 to 18) in Denmark who are sexually active.

\item All individuals in Poland who used the Internet in August, 2014.

\item All adult supporters of Greenpeace in Denmark.

\item All readers of the newspaper \textit{Jyllands-Posten}.

\item All local church parishes for all Christian religious denominations in the United States.

\end{enumerate}



\end{document}
