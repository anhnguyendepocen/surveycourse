\documentclass[12pt, a4]{article}
\usepackage[top=2cm, bottom=2cm, left=2cm, right=2cm]{geometry}
\usepackage{setspace}
\usepackage{mdwlist}

\title{Cognitive Interviewing Activity\vspace{-2em}}
\author{}
\date{}

\begin{document}

\maketitle

\onehalfspacing

\noindent The purpose of this activity is to practice cognitive interviewing. You will be using the questionnaire from the Spring 2014 Eurobarometer survey, which is a longitudinal data collection effort dating to 1973. Working with a partner, you should practice cognitive interviewing techniques. One person should serve as interviewer and one person should serve as respondent. You can take turns if you wish.

\vspace{1em}

\noindent For each question, the interviewer should read the question and then probe for points of clarification using any of the cognitive interviewing techniques we discussed in class and in the readings for this week. This might include:

\begin{itemize}
	\item Asking the respondent to ``think aloud'' as they attempt to answer each question before providing an answer
	\item Asking the respondent to describe their thought process after providing an answer
	\item Probing the respondent for whether they thought about specific considerations that are intended by the survey instrument (e.g., counting flatmates into estimates of household income)
	\item Recording any difficulties reported by the respondent or any clarifying questions asked by the respondent
	\item Probing whether the available response options made sense or whether the respondents' formulated answer did not fit any of the response options
	\item Any other techniques that seem appropriate
\end{itemize}

\noindent Be prepared to share your experiences as both interviewer and respondent with the class.



\vspace{1em}

\noindent 

\clearpage
\section*{Questionnaire}

\singlespacing

\begin{enumerate}\itemsep1.5em
	\item How would you judge the current situation in each of the following?
	\begin{enumerate}
		\item The situation of the Danish economy
		\item The situation of the European Union economy
		\item The financial situation of your household
		\item Your personal job situation
		\item The quality of life in your country
		\item The quality of life in the European Union
	\end{enumerate}
	\item What are your expectations for the next twelve months: will the next twelve months be better, worse, or the same, when it comes to\dots
		\begin{enumerate}
			\item The economic situation in Denmark
			\item The economic situation in the European Union
			\item The employment situation in your country
			\item Your personal job situation
		\end{enumerate}
	\item What do you think are the two most important issues facing your country at the moment?
	\item What do you think are the two most important issues you are facing at the moment?
	\item For each of the following media and institutions, please tell me if you tend to trust it or tend not to trust it.
		\begin{enumerate}
			\item The European Union
			\item The Danish parliament
			\item The Danish government
		\end{enumerate}
	\item Soe analysts say that the impact of the economic crisis on the job market has already reached its peak and things will recover little by little. Others, on the contrary, say that the worst is still to come. Which of the following two statements is closer to your opinions:
		\begin{enumerate}
			\item The impact of the crisis on jobs has already reached its peak
			\item The worst is still to come
		\end{enumerate}
	\item In your opinion, which of the following is best able to take effective actions against the effects of the financial and economic crisis? The European Union, the Danish government, the G20, the International Monetary Fund, the United States.
	\item Thinking about each of the following objectives to be reached by 2020 in the EU, would you say that it is too ambitious, about right, or too modest?
		\begin{enumerate}
			\item Three quarters of men and women between 20 and 64 years of age should have a job
			\item To increase energy efficiency in the EU by 20\% by 2020
			\item To increase the share of renewable energy in the EU by 20\% by 2020
			\item The share of funds invested in research and development should reach 3\% of the wealth produced in the EU each year
			\item To reduce EU greenhouse gas emissions by at least 20\% by 2020 compared to 1990
			\item The number of young people leaving school with no qualifications should fall to 10\%
			\item The number of Europeans living below the poverty line should be reduced by a quarter by 2020
			\item At least 40\% of the people aged 30 to 34 should have a higher education degree or diploma
		\end{enumerate}
	\item For each of the following statements, please tell me to what extent it corresponds or not to your own opinion.
		\begin{enumerate}
			\item You feel you are a citizen of the EU
			\item You know what your rights are as a citizen of the EU
			\item You would like to know more about your rights as a citizen of the EU
		\end{enumerate}
	\item Which of the following do you think is the most positive result of the EU?
		\begin{enumerate}
			\item The free movement of people, goods, and services within the EU
			\item Peace among the Member States of the EU
			\item The Euro
			\item Student exchange programs such as ERASMUS
			\item The political and diplomatic influence of the EU in the rest of the world
			\item The economic power of the EU
			\item The level of social welfare (healthcare, education, pensions) in the EU
			\item The common agricultural policy
		\end{enumerate}
	\item I am now going to read out different aspects of everyday life. For each, could you tell me if this aspect of your life is very satisfactory, fairly satisfactory, not very satisfactory, or not at all satisfactory?
		\begin{enumerate}
			\item Your house or flat
			\item The quality of life in the area where you live
			\item Your state of health
			\item Your standard of living
			\item The time you have available to do the things you want to do
		\end{enumerate}
\end{enumerate}

\end{document}
