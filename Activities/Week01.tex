\documentclass[a4, 12pt]{article}
\usepackage[top=2cm, bottom=2cm, left=2cm, right=2cm]{geometry}
\usepackage{bibentry}

%opening
\title{Survey Research Examples}
\author{}
\date{}

\begin{document}

\maketitle
\nobibliography*

\noindent For each article abstract, determine (1) the project's research question and (2) how a survey was useful in answering that question. In other words, how did the author(s) use a survey instrument as part of their overall research design?

\section*{Example 1}

In accordance with social exchange theory, prominent streams of management research emphasize the importance of reciprocal exchange relationships between organizations and their employees. When employees perceive themselves as supported by the organization, they reciprocate with increased work motivation. However, we do not know how this knowledge can be developed into management initiatives that increase public employees' perceived support, because severe endogeneity problems make it difficult to estimate the effect of organizational support on employee commitment outside the laboratory. We use a randomized field experiment involving more than 800 public employees to estimate the effect. We find no average effect of the organizational support treatment on the employees' perceived organizational support. Yet, a subgroup analysis shows a positive treatment effect when the employees' local front-line managers felt less supported prior to the intervention. We discuss the implications for theory and management practice.

\vspace{1em}\noindent From: \bibentry{JakobsenCalmar2013}

\section*{Example 2}

Interest groups pursue a wide range of policy goals. In their attempts to realize these goals, groups may lobby bureaucrats and politicians, approach the media and engage in protest activities. This article investigates the relation between the characteristics of policy goals and the strategies of influence utilized by interest groups. Policy goals are captured by four dimensions emphasizing: (i) the divisibility of goals, (ii) the degree of change sought, (iii) the type of interests pursued, and (iv) how technical goals are. The relevance of these dimensions and the effect of goals on influence strategies are tested in a survey of national Danish interest groups. The findings support the importance of group goals as determining strategy. Groups pursuing general interests mainly lobby parliament and the media, whereas groups with technically complicated goals lobby bureaucrats more intensively. The more divisible a goal a group is pursuing, the more actively it engages in all types of influence strategies.

\vspace{1em}\noindent From: \bibentry{BinderkrantzKroeyer2012}

\section*{Example 3}

A robust finding in the welfare state literature is that public support for the welfare state differs widely across countries. Yet recent research on the psychology of welfare support suggests that people everywhere form welfare opinions using psychological predispositions designed to regulate interpersonal help giving using cues regarding recipient effort. We argue that this implies that cross-national differences in welfare support emerge from mutable differences in stereotypes about recipient efforts rather than deep differences in psychological predispositions. Using free-association tasks and experiments embedded in large-scale, nationally representative surveys collected in the United States and Denmark, we test this argument by investigating the stability of opinion differences when faced with the presence and absence of cues about the deservingness of specific welfare recipients. Despite decades of exposure to different cultures and welfare institutions, two sentences of information can make welfare support across the U.S. and Scandinavian samples substantially and statistically indistinguishable.

\vspace{1em}\noindent From: \bibentry{AaroePetersen2014}

\section*{Example 4}

How are civilian attitudes toward combatants affected by wartime victimization? Are these effects conditional on which combatant inflicted the harm?We investigate the determinants of wartime civilian attitudes towards combatants using a survey experiment across 204 villages in five Pashtun-dominated provinces of Afghanistan---the heart of the Taliban insurgency. We use endorsement experiments to indirectly elicit truthful answers to sensitive questions about support for different combatants. We demonstrate that civilian attitudes are asymmetric in nature. Harm inflicted by the International Security Assistance Force (ISAF) is met with reduced support for ISAF and increased support for the Taliban, but Taliban-inflicted harm does not translate into greater ISAF support. We combine a multistage sampling design with hierarchical modeling to estimate ISAF and Taliban support at the individual, village, and district levels, permitting a more fine-grained analysis of wartime attitudes than previously possible.

\vspace{1em}\noindent From: \bibentry{LyallBlairImai2013}

\clearpage

\section*{Example 5}

The 1969 Vietnam draft lottery assigned numbers to birth dates in order to determine which young men would be called to fight in Vietnam. We exploit this natural experiment to examine how draft vulnerability influenced political attitudes. Data are from the Political Socialization Panel Study, which surveyed high school seniors from the class of 1965 before and after the national draft lottery was instituted. Males holding low lottery numbers became more antiwar, more liberal, and more Democratic in their voting compared to those whose high numbers protected them from the draft. They were also more likely than those with safe numbers to abandon the party identification that they had held as teenagers. Trace effects are found in reinterviews from the 1990s. Draft number effects exceed those for preadult party identification and are not mediated by military service. The results show how profoundly political attitudes can be transformed when public policies directly affect citizens' lives.

\vspace{1em}\noindent From: \bibentry{EriksonStoker2011}

\section*{Example 6}

Optimal jurisdiction size is a cornerstone of government design. A strong tradition in political thought argues that democracy thrives in smaller jurisdictions, but existing studies of the effects of jurisdiction size, mostly cross-sectional in nature, yield ambiguous results due to sorting effects and problems of endogeneity. We focus on internal political efficacy, a psychological condition that many see as necessary for high-quality participatory democracy. We identify a quasiexperiment, a large-scale municipal reform in Denmark, which allows us to estimate a causal effect of jurisdiction size on internal political efficacy. The reform, affecting some municipalities, but not all, was implemented by the central government, and resulted in exogenous, and substantial, changes in municipal population size. Based on survey data collected before and after the reform, we find, using various difference-in-difference and matching estimators, that jurisdiction size has a causal and sizeable detrimental effect on citizens' internal political efficacy.


\vspace{1em}\noindent From: \bibentry{LassenSerritzlew2011}

\bibliographystyle{plain}
\nobibliography{Week1}

\end{document}
